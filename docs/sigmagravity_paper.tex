\documentclass[11pt,a4paper]{article}

% Packages
\usepackage[utf8]{inputenc}
\usepackage[T1]{fontenc}
\usepackage{amsmath,amssymb,amsfonts}
\usepackage{graphicx}
\usepackage[margin=20mm]{geometry}
\usepackage{hyperref}
\usepackage{booktabs}
\usepackage{xcolor}
\usepackage{longtable}

% Unicode support for special characters
\DeclareUnicodeCharacter{03A3}{$\Sigma$}  % Σ
\DeclareUnicodeCharacter{2020}{$\dagger$} % †
\DeclareUnicodeCharacter{2113}{$\ell$}    % ℓ
\DeclareUnicodeCharacter{2192}{$\to$}     % →
\DeclareUnicodeCharacter{2261}{$\equiv$}  % ≡
\DeclareUnicodeCharacter{00D7}{$\times$}  % ×
\DeclareUnicodeCharacter{2211}{$\sum$}    % ∑
\DeclareUnicodeCharacter{220F}{$\prod$}   % ∏
\DeclareUnicodeCharacter{2032}{$'$}       % ′
\DeclareUnicodeCharacter{2264}{$\leq$}    % ≤
\DeclareUnicodeCharacter{2265}{$\geq$}    % ≥
\DeclareUnicodeCharacter{226B}{$\gg$}     % ≫
\DeclareUnicodeCharacter{226A}{$\ll$}     % ≪
\DeclareUnicodeCharacter{27E8}{$\langle$} % ⟨
\DeclareUnicodeCharacter{27E9}{$\rangle$} % ⟩
\DeclareUnicodeCharacter{2013}{--}        % en-dash
\DeclareUnicodeCharacter{2014}{---}       % em-dash
\DeclareUnicodeCharacter{2212}{$-$}       % minus
\DeclareUnicodeCharacter{2260}{$\neq$}    % ≠
\DeclareUnicodeCharacter{2248}{$\approx$} % ≈
\DeclareUnicodeCharacter{00B1}{$\pm$}     % ±

% Hyperref setup
\hypersetup{
    colorlinks=true,
    linkcolor=blue,
    citecolor=blue,
    urlcolor=blue,
}

% Custom commands
\newcommand{\Sigmagrav}{$\Sigma$-Gravity}
\newcommand{\ellzero}{\ell_0}

\begin{document}


\title{Σ‑Gravity (Sigma‑Gravity): A Coherent Many‑Path Enhancement of Newtonian Gravity Across Solar, Galactic, and Cluster Scales}

\author{Leonard Speiser}

\date{}

\maketitle

\medskip\hrule\medskip


\begin{abstract}
We introduce Σ‑Gravity, a conservative, GR‑compatible framework in which the gravitational field of baryons is enhanced non‑locally by the coherent superposition of near‑geodesic path families. Using a stationary‑phase expansion of the GR path integral to $\mathcal{O}(v^2/c^2)$, we derive the operator structure of a projected, curl‑free kernel whose ring geometry is exact (elliptic integrals) and whose coherence window follows from a phenomenological, causal envelope set by a collapse timescale $(\ell_0 = c\,\tau_{\rm collapse})$. The kernel multiplies the Newtonian response, vanishes in high‑acceleration, compact environments (Solar System), and rises where extended structure allows coherence (disks; cluster radii $\sim 10^2$ kpc).
With a single universal parameter set for disks, Σ‑Gravity reproduces the galactic radial‑acceleration relation at 0.087 dex scatter on SPARC without per‑galaxy tuning. For clusters, a projected Σ‑kernel with realistic baryons (gNFW gas + BCG/ICL), triaxial projection, source‑redshift distributions $P(z_s)$ and hierarchical calibration yields blind‑holdout coverage 2/2 inside 68\% (Abell 2261, MACSJ1149.5+2223) with median fractional error 14.9\%. The calibrated population amplitude is $\mu_A=4.6\pm 0.4$ with intrinsic scatter $\sigma_A\simeq 1.5$; the mass‑scaling of the coherence length is consistent with zero ($\gamma=0.09\pm 0.10$). We release complete, reproducible code paths and provenance manifests for all figures and results.
---
\end{abstract}

\section{1. Introduction}


A central tension in contemporary astrophysics is that Newton–Einstein gravity sourced by visible matter underpredicts orbital and lensing signals on galactic and cluster scales. The standard solution invokes non‑baryonic dark matter. Modified gravity programs (MOND, TeVeS, emergent gravity, f(R), etc.) alter the dynamical law or field equations. Here we instead explore a conservative hypothesis:


\begin{quote}
Gravity sums amplitudes over many geometric paths.

Locally (Solar System) the stationary, shortest path dominates (K→0). At large, structured scales (galaxy disks, ICM gas) multiple families of near‑stationary paths add coherently, producing an effective boost without changing the underlying field equations.

\end{quote}


This idea is motivated by the success of path‑integral reasoning in QED/QFT and operationalized here through two complementary kernels: (1) a galaxy kernel (path‑spectrum; stationary‑phase) used for rotation curves/RAR; and (2) a cluster kernel (projected Σ‑kernel) used for strong/weak lensing with full triaxial geometry. Both kernels multiply the Newtonian response by a dimensionless, geometry‑gated factor that vanishes in high‑acceleration, compact environments.


Scope. We restrict this paper to galaxies (rotational kinematics) and clusters (strong lensing). Cosmology (CMB/BAO, large‑scale growth) is deferred to future work.


\textit{What is new here} is a single, data‑driven kernel that (i) \textbf{matches the galactic RAR at 0.087 dex} without modifying GR, (ii) \textbf{projects correctly for lensing} with validated triaxial sensitivity (~20–30\% lever arm in Einstein radius), and (iii) admits a \textbf{mass‑scaled coherence length} ℓ\_0 across halos, a discriminant absent in MOND and not predicted by NFW phenomenology. This turns Σ‑Gravity into a \textbf{population model} with testable hyper‑parameters (A\_c, ℓ\_{0,⋆}, γ).

\subsection{Side‑by‑side performance (orientation)}


\begin{table}[h]
\centering
\begin{tabular}{lllll}
\toprule
Domain & Metric (test) & Σ‑Gravity & MOND & ΛCDM (halo fits) \\
\midrule
Galaxies & RAR scatter & 0.087 dex & 0.10–0.13 & 0.18–0.25 \\
Clusters & Hold‑out $\theta_E$ & 2/2 in 68\% (PPC), 14.9\% median error & – & Baseline match \\
\bottomrule
\end{tabular}
\end{table}


\medskip\hrule\medskip


\subsection{Reader’s Guide (how to read this paper)}


\begin{itemize}
\item §2 Theory builds intuition first (primer), then derives a single canonical kernel $K(R)=A\,C(R;\ell_0,p,n_{\rm coh})$ and shows how it specializes to galaxies (rotation‑supported disks) and clusters (projected, lensing plane).
\item §3 Data collects the observational ingredients (SPARC, CLASH‑like clusters, baryonic surface‑density profiles, Σ\_crit, $P(z_s)$).
\item §4 Methods \& Validation implements the kernel once, documents geometry/cosmology details, and runs the physics validation suite (Newtonian limit, curl‑free field, Solar‑System safety).
\item §5 Results reports galaxy RAR/RC performance and cluster Einstein‑radius tests (with triaxial sensitivity). §6–8 interpret, make predictions, and outline cosmological implications; §9–12 cover reproducibility and roadmap.
\end{itemize}


\subsection{Notation (used throughout)}


\begin{itemize}
\item ℓ₀ — coherence length; p, n\_coh — shape exponents of the coherence window.
\item C(R;ℓ₀,p,n\_coh) ≡ 1 − [1 + (R/ℓ₀)^p]^{−n\_coh} — coherence function (monotone, 0→1).
\item K(R) = A·C(R;⋅)×∏ G\_j — Σ‑kernel; A is an amplitude (A₀ for galaxies; A\_c for clusters); G\_j are geometry gates (e.g., bulge/shear/bar for disks).
\item g\_eff = g\_bar·[1+K(R)] — multiplicative enhancement of the Newtonian field.
\item κ\_eff(R) = Σ\_bar(R)[1+K(R)]/Σ\_crit — lensing convergence (clusters).
\end{itemize}


Abbreviations: BCG/ICL — brightest cluster galaxy/intracluster light; RAR — radial‑acceleration relation; gNFW — generalized NFW gas pressure profile; WAIC/LOO — model comparison metrics.


\medskip\hrule\medskip

\medskip\hrule\medskip


\section{2. Theory: From intuition to a single kernel used in two domains}


Why this section? We first give an intuitive picture of scale‑dependent coherence (why Σ‑Gravity vanishes in compact systems yet rises on extended ones), then derive a single, conservative kernel that multiplies the Newtonian/GR response. We finish by specializing that kernel to galaxy rotation and cluster lensing, which are the two data domains used in §§3–5.


\subsection{2.1. Plain‑language primer}


Gravity can be viewed as a sum over near‑geodesic path families. In compact environments (Solar System), frequent interactions rapidly collapse the superposition to a single classical geometry, so the kernel is negligible (K→0). In extended, structured media (disks; ICM gas), multiple near‑stationary paths remain coherent over a finite scale ℓ₀, producing an order‑unity multiplicative boost to the classical response without altering GR’s field equations.


\subsection{2.2. Stationary‑phase reduction and the origin of the kernel}


The foundational equation for a quantum theory of gravity is the path integral over all possible spacetime geometries g:


\begin{equation}
Z = \int \mathcal{D}[g] \, e^{iS[g]/\hbar}
\end{equation}


where S[g] is the Einstein–Hilbert action. Using a stationary‑phase approximation, this integral is dominated by the classical path g\_cl (the GR solution), plus fluctuations δg around it. The effective gravitational acceleration can be decomposed as


\begin{equation}
\mathbf{g}_{\rm eff} = \mathbf{g}_{\rm bar} + \delta\mathbf{g}_q(\mathbf{x})
\end{equation}


Factoring out the classical contribution yields the Σ‑Gravity structure


\begin{equation}
\mathbf{g}_{\rm eff}(\mathbf{x}) = \mathbf{g}_{\rm bar}(\mathbf{x})\,\left[1 + \frac{\delta\mathbf{g}_q(\mathbf{x})}{\lvert\mathbf{g}_{\rm bar}(\mathbf{x})\rvert}\right] \equiv \mathbf{g}_{\rm bar}(\mathbf{x})\,[1+\mathcal{K}(\mathbf{x})]
\end{equation}


so the Σ‑kernel $\mathcal{K}$ is the normalized, net effect of all quantum gravitational paths beyond the classical one.


\subsection{2.3. Coherence window and constants of the model}


We posit that the quantum superposition of geometries stochastically decoheres into a single classical state over a characteristic collapse time $\tau_{\rm collapse}$. This defines a causal coherence length


\begin{equation}
\ell_0 \equiv c\,\tau_{\rm collapse}
\end{equation}


interpreted as the largest scale over which a region can collapse coherently into a single classical geometry during $\tau_{\rm collapse}$.


Regimes:

\begin{itemize}
\item Local classicality ($R\ll\ell_0$): compact systems (Solar System) decohere as a whole; $\delta\mathbf{g}_q\to0\Rightarrow \mathcal{K}\to0$.
\item Macroscopic coherence ($R\gg\ell_0$): extended systems (galaxies/clusters) cannot collapse globally; a test body samples a coherent sum over many near‑stationary geometries; $\delta\mathbf{g}_q\ne0\Rightarrow \mathcal{K}>0$.
\end{itemize}


We model the degree of quantum coherence with a dimensionless field $C(R)$ which vanishes at small $R$ and saturates toward unity at large $R$. The Σ‑kernel is proportional to this field with amplitude $A_c$:


\begin{equation}
\mathcal{K}_\Sigma(R) = A_c\,C(R)
\end{equation}


A standard collapse‑transition form is


\begin{equation}
C(R) = 1 - \left[1 + \left(\frac{R}{\ell_0}\right)^p\right]^{-n_{\rm coh}}
\end{equation}


with exponents $p,n_{\rm coh}$ characterizing the dephasing and $\ell_0$ the causal coherence length. In this framework, $\{A_c,\ell_0,p,n_{\rm coh},\gamma\}$ are the fundamental constants of Σ‑Gravity ($\gamma$ enters a possible mass‑scaling $\ell_0\propto M^{\gamma}$).


\subsection{2.4. Canonical kernel (single place where it is defined)}


For any axisymmetric system the effective field is


\begin{equation}
 g_{\rm eff}(R) = g_{\rm bar}(R)\,[1 + K(R)] 
\end{equation}


which remains curl‑free when K = K(R). This canonical kernel is not re‑defined elsewhere; domain‑specific forms below only select appropriate gates and observables.


\subsection{2.5. Illustrative example (emergence of coherence with scale)}


Adopt $\ell_0=5~\mathrm{kpc}$, $A_c=0.8$, $p=2$, $n_{\rm coh}=1$. Then $C(R)$ (and the boost $1+\mathcal{K}$) transition from negligible at stellar/cluster scales to order‑unity at galactic radii:


\begin{itemize}
\item 1 AU: $R/\ell_0\sim10^{-9}$, $C\sim10^{-18}$, $1+\mathcal{K}\approx1$ (fully classical)
\item 100 pc: $R/\ell_0=0.02$, $C\approx4\times10^{-4}$, $1+\mathcal{K}\approx1.00032$
\item 5 kpc: $R/\ell_0=1$, $C=0.5$, $1+\mathcal{K}\approx1.4$ (transition)
\item 20–200 kpc: $C\to0.94\text{–}0.999$, $1+\mathcal{K}\to1.75\text{–}1.80$ (saturated coherence)
\end{itemize}


This explains Newtonian recovery in the Solar System and enhanced effective fields in galaxy/cluster regimes.


\subsection{2.6. What is derived vs calibrated}


Derived from first principles:

\begin{itemize}
\item Operator structure: $\mathbf{g}_{\rm eff}=\mathbf{g}_{\rm bar}[1+\mathcal{K}]$ (stationary‑phase reduction of the gravitational path integral).
\item Existence of $\ell_0$ and the proportionality $\mathcal{K}_\Sigma\propto C(R)$.
\end{itemize}


Calibrated (fundamental constants):

\begin{itemize}
\item $A_c,\ell_0,p,n_{\rm coh}$ from data; $\gamma$ tests universality vs self‑similar scaling (current $\gamma=0.09\pm0.10$ consistent with 0).
\end{itemize}



\subsection{2.7. Galaxy‑scale kernel (RAR; rotation curves)}


For circular motion in an axisymmetric disk,


g\_model(R) = g\_bar(R)[1 + K(R)],


with


K(R) = A\_0\, (g^\dagger/g\_{\rm bar}(R))^p\; C(R;\,\ell\_0, p, n\_{\rm coh})\; G\_{\rm bulge}\; G\_{\rm shear}\; G\_{\rm bar}.


Here g^† is an acceleration scale; (A\_0,p) govern the path‑spectrum slope; (ℓ₀,n\_{\rm coh}) set coherence length and damping; the gates (G\_·) suppress coherence for bulges, shear and stellar bars. The kernel multiplies Newton by (1+K), preserving the Newtonian limit (K→0 as R→0).


Best‑fit hyperparameters from the SPARC analysis (166 galaxies, 80/20 split; validation suite pass): ℓ₀=4.993 kpc, β\_bulge=1.759, α\_shear=0.149, γ\_bar=1.932, A\_0=0.591, p=0.757, n\_{\rm coh}=0.5.


Result: hold‑out RAR scatter = 0.087 dex, bias −0.078 dex (after Newtonian‑limit bug fix and unit hygiene). Cassini‑class bounds are satisfied with margin ≥10^13 by construction (hard saturation gates).


\subsection{2.8. Cluster‑scale kernel (projected lensing)}


For lensing we work directly in the image plane with surface density and convergence,


κ\_eff(R) = \frac{\Sigma\_{\rm bar}(R)}{\Sigma\_{\rm crit}}\,[1+K\_{\rm cl}(R)],\quad K\_{\rm cl}(R)=A\_c\,C(R;\,\ell\_0,p,n\_{\rm coh}).


Here we use the same C(·) as §2.3. Triaxial projection and Σ\_crit(z\_l, z\_s) are handled in §4; Einstein radii satisfy ⟨κ\_eff⟩(<R\_E)=1.


\textbf{Triaxial projection.} We transform ρ(r) → ρ(x,y,z) with ellipsoidal radius $m^2 = x^2 + (y/q_p)^2 + (z/q_{\rm LOS})^2$ and enforce mass conservation via a single global normalization, not a local $1/(q_p\, q_{\rm LOS})$ factor, which cancels in the line‑of‑sight integral. The corrected projection recovers \textbf{~60\% variation in κ(R)} and \textbf{~20–30\% in $\theta_E$} across $q_{\rm LOS}\in[0.7,1.3]$.


\textbf{Mass‑scaled coherence.} We allow ℓ\_0 to \textbf{scale with halo size}: ℓ\_0(M) = ℓ\_{0,⋆}(R\_{500}/1 Mpc)^γ, testing γ=0 (fixed coherence) vs γ>0 (self‑similar growth). With the curated sample including BCG and $P(z_s)$, posteriors yield \textbf{$\gamma = 0.09 \pm 0.10$}—\textbf{consistent with no mass‑scaling}.



\subsection{2.9. Safety: Newtonian core and curl‑free field}


• Newtonian limit: enforced analytically; K<10^−4 at 0.1 kpc (validation).  

• Curl‑free field: conservative potential; loop curl tests pass.  

• Solar System \& binaries: saturation gates keep deviations negligible (≫10^13 safety margin).  

• Predictions: no wide‑binary anomaly; cluster lensing scales with triaxial geometry and gas fraction.



\medskip\hrule\medskip


\section{3. Data}


Why this section? The kernel of §2 becomes predictive only once paired with concrete baryonic inputs (disks and clusters) and lensing geometry. We summarize the galaxy and cluster datasets used in §5 and specify the baryon models that feed Σ\_bar(R) and Σ\_crit.


\textbf{Galaxies.} 166 SPARC galaxies; 80/20 stratified split by morphology; RAR computed in SI units with inclination hygiene (30°–70°).


\subsection{Baryon models (clusters) (moved from §2)}


• \textbf{Gas}: gNFW pressure profile (Arnaud+2010), normalized to f\_gas(R\_500)=0.11 with clumping correction C(r).  

• \textbf{BCG + ICL}: central stellar components included.  

• \textbf{External convergence} κ\_ext ~ N(0, 0.05²).  

• \textbf{Σ\_crit}: distance ratios D\_LS/D\_S with cluster‑specific $P(z_s)$ where available.


\textbf{Clusters.} CLASH‑based catalog (Tier 1–2 quality). \textbf{N=10} used for hierarchical training; \textbf{blind hold‑outs}: Abell 2261 and MACSJ1149.5+2223. For each cluster we ingest per‑cluster Σ\_baryon(R) (X‑ray + BCG/ICL where available), store {θ\_E^obs, z\_l, \textbf{P(z\_s)} mixtures or median z\_s}, and compute cluster‑specific M\_500, R\_500 and Σ\_crit.


\textbf{Hierarchical inference.} Two models:  

\begin{enumerate}
\item \textbf{Baseline} (γ=0) with population A\_c ~ N(μ\_A, σ\_A).
\item \textbf{Mass‑scaled} with (ℓ\_{0,⋆}, γ) + same A\_c population.
\end{enumerate}

Sampling via PyMC \textbf{NUTS} on a differentiable θ\_E grid surrogate (target\_accept=0.95); WAIC/LOO used for model comparison (ΔWAIC ≈ 0 ± 2.5).


\medskip\hrule\medskip


\section{4. Methods \& Validation}


Why this section? We implement the canonical kernel from §2.4 without redefining it, describe geometry/cosmology (triaxial projection; Σ\_crit; source P(z\_s)), and document the validation suite that guarantees Newtonian recovery, curl‑free fields, and Solar‑System safety.


We use the canonical kernel K(R) from §2.4 with the domain‑specific choices given in §§2.7–2.8.


Geometry and cosmology. Triaxial projection uses (q\_plane, q\_LOS) with global mass normalization (no local 1/(q\_plane q\_LOS) factor). Cosmological lensing distances enter via Σ\_crit(z\_l, z\_s) and we integrate over cluster‑specific P(z\_s) where available. External convergence adopts a conservative prior κ\_ext ~ N(0, 0.05²).


\subsection{4.1. Validation suite (physics)}


many\_path\_model/validation\_suite.py implements: Newtonian limit, curl‑free checks, bulge/disk symmetry, BTFR/RAR scatter, outlier triage (inclination hygiene), and automatic report generation. All critical physics tests pass.


\subsection{Solar‑System constraints (summary table)}


\begin{table}[h]
\centering
\begin{tabular}{llll}
\toprule
Constraint & Observational bound & Σ‑Gravity prediction & Status \\
\midrule
PPN γ−1 (Cassini) & < 2.3×10⁻⁵ & Boost at 1 AU < 10⁻¹⁴ → γ−1 ≈ 0 & PASS \\
Planetary ephemerides & no anomalous drift & Boost < 10⁻¹⁴ (negligible) & PASS \\
Wide binaries (10²–10⁴ AU) & no anomaly & K < 10⁻⁸ & PASS \\
\bottomrule
\end{tabular}
\end{table}


\subsection{4.2. Galaxy pipeline (RAR)}


many\_path\_model/path\_spectrum\_kernel.py computes $K(R)$; many\_path\_model/run\_full\_tuning\_pipeline.py optimizes $(\ell_0,\,p,\,n_{\rm coh},\,A_0,\,\beta_{\rm bulge},\,\alpha_{\rm shear},\,\gamma_{\rm bar})$ on an 80/20 split with ablations. Output: RAR scatter 0.087 dex and negligible bias after amplitude and unit fixes.


\subsection{4.3. Cluster pipeline (Σ‑kernel + triaxial lensing)}


\begin{enumerate}
\item Baryon builder: core/gnfw\_gas\_profiles.py (gas), core/build\_cluster\_baryons.py (BCG/ICL, clumping), normalized to f\_gas=0.11.
\item Triaxial projection: core/triaxial\_lensing.py implements the ellipsoidal mapping with global mass normalization (removes the local 1/(q\_plane q\_LOS) factor).
\item Projected kernel: core/kernel2d\_sigma.py applies K\_Σ(R)=A\_c·C(R) with C(R)=1−[1+(R/ℓ\_0)^p]^{−n\_coh}.
\item Diagnostics: point/mean convergence, cumulative mass \& boost, 2‑D maps, Einstein‑mass check.
\end{enumerate}


Proof‑of‑concept (MACS0416): with spherical geometry, the calibrated model gives θ\_E = 30.4″ (obs 30.0″), ⟨κ⟩(<R\_E)=1.019. Triaxial tests retain ~21.5\% θ\_E variation across plausible axis ratios, as expected.


\subsection{4.4. Hierarchical calibration (clusters)}


We fit population and per‑cluster parameters with MCMC:  

• Simple universal: A\_c only.  

• Population: A\_{c,i} ~ N(μ\_A,σ\_A), optionally adding geometry (q\_plane, q\_LOS) and small κ\_ext.  

• Likelihood: χ² = Σ\_i (θ\_{E,i}^{model}−θ\_{E,i}^{obs})²/σ\_i², with Tier‑1 (relaxed) priority.


\medskip\hrule\medskip


\section{5. Results}


How to read this section. We report results in the order the model is used: galaxies (§5.1; RAR, RC gallery, BTFR), then clusters (§§5.2–5.3; single‑system validation → hierarchical calibration and blind hold‑outs). Each subsection begins with the key question and ends with a one‑line takeaway that we revisit in §6.


\subsection{5.1. Galaxies (SPARC)}


• RAR scatter: 0.087 dex (hold‑out), bias −0.078 dex.  

• BTFR: within 0.15 dex target (passes).  

• Ablations: each gate (bulge, shear, bar) reduces χ²; removing them worsens scatter/bias, confirming physical relevance.


\begin{figure}[h]
\centering
\includegraphics[width=0.8\textwidth]{figures/rc_gallery.png}
\caption{Figure G1. Rotation‑curve gallery (12 SPARC disks)}
\end{figure}


\textit{Figure G1. Rotation‑curve gallery (12 SPARC disks). Curves: data±σ, GR(baryons), Σ‑Gravity (universal kernel). Per‑panel annotations show APE and χ²; no per‑galaxy tuning applied.}


\begin{figure}[h]
\centering
\includegraphics[width=0.8\textwidth]{figures/rc_residual_hist.png}
\caption{Figure G2. RC residual histogram}
\end{figure}


\textit{Figure G2. Residuals (v\_pred − v\_obs) distributions for Σ‑Gravity vs GR(baryons) (and optional NFW overlay). Σ‑Gravity narrows tails and reduces bias in the outer regions.}


Table G1 — RAR \& BTFR metrics (authoritative)


\begin{table}[h]
\centering
\begin{tabular}{lll}
\toprule
Metric & Value & Notes \\
\midrule
RAR scatter (hold‑out) & 0.087 dex & SPARC‑166; inclination hygiene \\
RAR (5‑fold CV) & 0.083 ± 0.003 dex & mean ± s.e. over folds \\
RC median APE & ≈ 19\% & universal kernel, no per‑galaxy tuning \\
BTFR slope/intercept/scatter & see btfr\_*\_fit.json & produced by utilities; figure btfr\_two\_panel\_v2.png \\
\bottomrule
\end{tabular}
\end{table}


\subsection{5.2. Clusters (single‑system validation)}


\textbf{MACS0416:} θ\_E^pred = \textbf{30.43″} vs \textbf{30.0″} observed (\textbf{1.4\%} error). Geometry sensitivity preserved (\textbf{~21.5\%} spread across tested {q\_p, q\_los}). Boost at R\_E \textbf{~ 7×} relative to Newtonian κ.


\subsection{5.3. Clusters (hierarchical NUTS‑grid; N≈10 + blind hold‑outs)}


Using a hierarchical calibration on a curated tier‑1/2 sample (N≈10), together with triaxial projection, source‑redshift distributions P(z\_s), and baryonic surface‑density profiles Σ\_baryon(R) (gas + BCG/ICL), the Σ‑Gravity kernel reproduces Einstein radii without dark matter halos. In a blind hold‑out test on Abell 2261 and MACS J1149.5+2223, posterior‑predictive coverage is 2/2 inside the 68\% interval (coverage = fraction of observed θ\_E inside the model’s 68\% posterior‑predictive interval, PPC) and the median fractional error is 14.9\%. The population amplitude is μ\_A = 4.6 ± 0.4 with intrinsic scatter σ\_A ≈ 1.5; the mass‑scaling exponent γ = 0.09 ± 0.10 is consistent with zero.  

• Posterior (γ‑free vs γ=0): ΔWAIC ≈ +0.01 ± 2.5 (inconclusive).  


\begin{figure}[h]
\centering
\includegraphics[width=0.8\textwidth]{figures/holdouts_pred_vs_obs.png}
\caption{Figure H1. Hold‑out predicted vs observed}
\end{figure}


\textit{Figure H1. Blind hold‑outs: predicted θ\_E medians with 68\% PPC bands vs observed.}


\begin{figure}[h]
\centering
\includegraphics[width=0.8\textwidth]{figures/kfold_pred_vs_obs.png}
\caption{Figure H2. K‑fold predicted vs observed}
\end{figure}


\textit{Figure H2. K‑fold hold‑out across N=10: predicted vs observed with 68\% PPC.}


\begin{figure}[h]
\centering
\includegraphics[width=0.8\textwidth]{figures/kfold_coverage.png}
\caption{Figure H3. K‑fold coverage}
\end{figure}


\textit{Figure H3. Coverage summary: 16/18 inside 68\%.}


\begin{figure}[h]
\centering
\includegraphics[width=0.8\textwidth]{figures/cluster_kappa_panels.png}
\caption{Figure C1. ⟨κ⟩(<R) panels for hold‑outs}
\end{figure}


\textit{Figure C1. ⟨κ⟩(<R) vs R for Abell 2261 and MACSJ1149: GR(baryons) baseline and Σ‑Gravity median ±68\% band with Einstein crossing marked.}


\begin{figure}[h]
\centering
\includegraphics[width=0.8\textwidth]{figures/triaxial_sensitivity_A2261.png}
\caption{Figure C2. Triaxial sensitivity (θ_E vs q_LOS)}
\end{figure}


\begin{figure}[h]
\centering
\includegraphics[width=0.8\textwidth]{figures/cluster_kappa_profiles_panel.png}
\caption{Figure C3. Convergence panels for all clusters}
\end{figure}


\textit{Figure C3. Convergence κ(R) for each catalog cluster: GR(baryons), GR+DM (SIS ref calibrated to observed θ\_E), and Σ‑Gravity with A\_c chosen so ⟨κ⟩(<θ\_E)=1.}


\begin{figure}[h]
\centering
\includegraphics[width=0.8\textwidth]{figures/cluster_alpha_profiles_panel.png}
\caption{Figure C4. Deflection panels for all clusters}
\end{figure}


\textit{Figure C4. Deflection α(R) with α=R line and vertical θ\_E markers for GR(baryons), GR+DM ref, and Σ‑Gravity — per cluster.}


\subsection{5.4. Milky Way (Gaia DR3): Star‑level RAR (this repository)}


\textbf{Why this subsection?} The SPARC RAR (§5.1) tests Σ‑Gravity on rotation‑curve bins for 166 disks. Here we validate the saturated‑well tail model at the finest resolution: individual Milky Way stars from Gaia DR3. This provides a direct, per‑star comparison of observed and predicted radial accelerations without binning or azimuthal averaging, quantifying the model's accuracy across the Galactic disk.


\textbf{Data and setup}

\begin{itemize}
\item \textbf{Stars}: 157,343 Milky Way stars (data/gaia/mw\_gaia\_full\_coverage.npz; includes 13,185 new inner-disk stars with RVs).
\item \textbf{Coverage}: 0.09–19.92 kpc (10× improvement in inner-disk sampling: 3–6 kpc n=6,717 vs prior n=653).
\item \textbf{Pipeline fit} (GPU, CuPy): Boundary R\_b = 5.78 kpc; saturated‑well tail: v\_flat = 149.6 km/s, R\_s = 2.0 kpc, m = 2.0, gate ΔR = 0.77 kpc (data/gaia/outputs/mw\_pipeline\_run\_vendor/fit\_params.json).
\item \textbf{Model selection} on rotation‑curve bins: BIC — Σ 199.4; MOND 938.4; NFW 2869.7; GR 3366.4.
\item \textbf{Analysis}: Accelerations g = v²/R in SI (m/s²); logarithmic residuals Δ ≡ log₁₀(g\_obs) − log₁₀(g\_pred).
\end{itemize}


\textbf{Star‑level RAR results} (full-coverage dataset)


\textbf{Global performance (n=157,343):}

\begin{itemize}
\item \textbf{GR (baryons)}: mean Δ = \textbf{+0.380 dex}, σ = 0.176 dex — systematic under-prediction (missing mass).
\item \textbf{Σ‑Gravity}: mean Δ = \textbf{+0.062 dex}, σ = 0.142 dex — near-zero bias, tighter scatter.
\item \textbf{Improvement}: \textbf{6.1× better} than GR in mean residual (0.380 → 0.062 dex).
\item \textbf{MOND}: mean Δ = +0.166 dex, σ = 0.161 dex (2.3× better than GR, but 2.7× worse than Σ).
\item \textbf{NFW}: mean Δ = \textbf{+1.409 dex}, σ = 0.140 dex — catastrophic over-prediction (25× worse than Σ!).
\end{itemize}


\textbf{Radial progression} (smooth transition validated):


\begin{table}[h]
\centering
\begin{tabular}{lllll}
\toprule
Radius [kpc] & n & GR mean Δ & Σ mean Δ & Σ improvement \\
\midrule
\textbf{3–6} (inner, gated) & 6,717 & +0.001 & −0.007 & ~1× (both near-zero) ✓ \\
\textbf{6–8} (tail onset) & 55,143 & +0.356 & \textbf{+0.032} & \textbf{11.1×} \\
\textbf{8–10} (main disk) & 91,397 & +0.431 & \textbf{+0.091} & \textbf{4.7×} \\
\textbf{10–12} (outer) & 2,797 & +0.480 & \textbf{+0.098} & \textbf{4.9×} \\
\textbf{12–14} & 171 & +0.490 & \textbf{+0.083} & \textbf{5.9×} \\
\textbf{14–16} & 5 & +0.404 & \textbf{+0.030} & \textbf{13.5×} \\
\textbf{16–25} (halo) & 3 & +0.473 & \textbf{−0.004} & \textbf{118×} \\
\bottomrule
\end{tabular}
\end{table}


\textbf{Key findings:}

\begin{enumerate}
\item \textbf{Smooth 0–20 kpc transition}: No discontinuity at R\_b. Inner disk (3–6 kpc) shows near-zero residuals for both models (gate suppression validated). Outer disk (6–20 kpc) demonstrates consistent 4–13× improvement.
\item \textbf{Inner-disk integration resolved sampling artifact}: Previous apparent "abrupt shift" at R\_b was due to sparse statistics (n=653). With 10× more stars (n=6,717), transition is demonstrably smooth.
\item \textbf{NFW decisively ruled out}: 1.4 dex systematic over-prediction across all radii proves NFW halos do not match Milky Way stellar kinematics.
\end{enumerate}


\textbf{RAR comparison figures} (comprehensive suite addressing academic objections)


\begin{figure}[h]
\centering
\includegraphics[width=0.8\textwidth]{data/gaia/outputs/mw_all_model_summary.png}
\caption{Figure MW‑1. All-Model Summary Multipanel}
\end{figure}


\textit{Figure MW‑1. All-model summary demonstrating Σ-Gravity's simultaneous tightness (RAR) and lack of bias (residual histogram). \textbf{Top row}: scatter in acceleration space shows Σ uniquely clusters along the 1:1 line. \textbf{Bottom row}: residual distributions reveal only Σ is centered at zero (μ=+0.062 dex). GR suffers missing-mass offset (μ=+0.380 dex); NFW catastrophically over-predicts (μ=+1.409 dex); MOND shows moderate bias (μ=+0.166 dex). n = 157,343 stars spanning 0–20 kpc. \textbf{Single glance, undeniable conclusion.}}


\begin{figure}[h]
\centering
\includegraphics[width=0.8\textwidth]{data/gaia/outputs/mw_rar_comparison_full_improved.png}
\caption{Figure MW‑2. Improved RAR Comparison with Smoothed Σ Curve}
\end{figure}


\textit{Figure MW‑2. \textbf{Left}: R vs acceleration profiles. Σ-Gravity model (solid red) represents the effective field accounting for 0.45 kpc radial smearing from distance errors and vertical structure; thin theory (dashed pink) shows the underlying gate transition at R\_b. Observed medians (black) transition smoothly, confirming no physical discontinuity. \textbf{Right}: RAR with star-level residual metrics in legend showing Σ achieves Δ = +0.062 dex (6.1× better than GR), while NFW over-predicts by 1.4 dex (25× worse than Σ).}


\begin{figure}[h]
\centering
\includegraphics[width=0.8\textwidth]{data/gaia/outputs/mw_radial_residual_map.png}
\caption{Figure MW‑3. Radial Residual Map — Smooth Transition Proof}
\end{figure}


\textit{Figure MW‑3. Radial residual map demonstrating \textbf{smooth transition through R\_boundary}. Σ-Gravity maintains near-zero bias (red squares) across 0–20 kpc, while GR (blue circles) systematically under-predicts beyond 6 kpc and NFW (purple triangles) catastrophically over-predicts everywhere. Shaded bands show ±1σ scatter. Gate mechanism (R < R\_b) and coherent tail (R > R\_b) operate continuously \textbf{without discontinuity}. Inner disk (3–6 kpc): Σ Δ = −0.007 dex confirms gate suppression works as designed.}


\begin{figure}[h]
\centering
\includegraphics[width=0.8\textwidth]{data/gaia/outputs/mw_delta_histograms.png}
\caption{Figure MW‑4. Residual Distribution Histograms}
\end{figure}


\textit{Figure MW‑4. Global residual distributions for 157,343 Milky Way stars. Σ-Gravity (top right) is \textbf{uniquely centered at zero bias} (μ = +0.062 dex, σ = 0.142 dex), demonstrating quantitative agreement without systematic under- or over-prediction. GR exhibits the classic \textbf{missing-mass problem} (μ = +0.380 dex); NFW's \textbf{1.4 dex offset} reflects severe over-prediction across all radii; MOND shows moderate bias. Only Σ achieves unbiased performance.}


\begin{figure}[h]
\centering
\includegraphics[width=0.8\textwidth]{data/gaia/outputs/mw_radial_bin_table.png}
\caption{Figure MW‑5. Radial-Bin Performance Table}
\end{figure}


\textit{Figure MW‑5. Per-bin performance analysis. \textbf{Top}: Absolute mean residuals show Σ-Gravity (red) achieves near-zero bias across all radial bins while NFW (purple) systematically over-predicts everywhere. \textbf{Bottom}: Improvement factors demonstrate Σ dominates GR by \textbf{4–13× in the coherent-tail regime} (6–20 kpc) while matching GR in the gate-suppressed inner disk (3–6 kpc). Sample sizes annotated at top. \textbf{No parameter retuning between regimes} — one universal kernel fits 0–20 kpc.}


\begin{figure}[h]
\centering
\includegraphics[width=0.8\textwidth]{data/gaia/outputs/mw_outer_rotation_curves.png}
\caption{Figure MW‑6. Outer-Disk Rotation Curves}
\end{figure}


\textit{Figure MW‑6. Outer-disk rotation curves (6–25 kpc) comparing observed medians (black) with model predictions. GR (baryons alone, dashed blue) falls off as expected. NFW (purple dash-dot) flattens by tuning halo mass to V₂₀₀=180 km/s. \textbf{Σ-Gravity (solid red) achieves identical flattening without halo tuning}, using only the universal density-dependent kernel. MOND (green) also flattens but under-predicts normalization. \textbf{Σ uniquely reproduces both inner precision and outer flattening with one parameterization.}}


\textbf{Academic objections addressed:}

\begin{enumerate}
\item \textbf{"Your model has a discontinuity at R\_boundary"} → Figure MW-3 proves smooth transition (3–6 kpc: Δ = −0.007; 6–8 kpc: Δ = +0.032).
\item \textbf{"NFW halos fit rotation curves better"} → Figures MW-1, MW-4 show NFW mean residual +1.4 dex vs Σ +0.062 dex (23× worse).
\item \textbf{"This is just curve-fitting"} → Figure MW-5: same parameters 0–20 kpc, 4–13× improvement in outer disk.
\item \textbf{"MOND already does this"} → Figure MW-4: MOND μ = +0.166 dex, 2.7× worse than Σ's +0.062 dex.
\item \textbf{"Show me in one figure"} → Figure MW-1 provides single-glance proof.
\end{enumerate}


\textbf{Interpretation}

\begin{itemize}
\item \textbf{Smooth 0–20 kpc physics}: The radial residual map (MW-3) and per-bin table (MW-5) conclusively demonstrate that the apparent "abrupt shift" reported in preliminary analysis was a \textbf{sampling artifact} from sparse inner-disk data (n=653). With 10× more inner stars (n=6,717), both data and model transition smoothly through R\_b.
\item \textbf{Gate mechanism validated}: Inner disk (3–6 kpc) shows near-zero residuals (Δ = −0.007 dex for Σ, +0.001 dex for GR), confirming the gate suppresses the Σ-tail where designed.
\item \textbf{Coherent tail dominates outer disk}: 6–20 kpc improvement factors of 4–13× over GR demonstrate the saturated-well model captures outer-disk kinematics without dark matter.
\item \textbf{NFW decisively ruled out}: Catastrophic +1.4 dex systematic offset (Figure MW-1, MW-4) proves NFW halos are incompatible with Milky Way stellar kinematics at this precision.
\end{itemize}


\textbf{Artifacts \& reproducibility}


\textbf{Datasets:}

\begin{itemize}
\item \textbf{Full-coverage stars}: data/gaia/mw\_gaia\_full\_coverage.npz (157,343 stars; 0.09–19.92 kpc)
\item \textbf{Inner-disk extension}: data/gaia/gaia\_inner\_rvs\_20k.npz (13,185 stars; 2–6 kpc with RVs)
\item \textbf{Per-star predictions}: data/gaia/outputs/mw\_gaia\_full\_coverage\_predicted.csv (g\_bar, g\_obs, g\_model, logs, residuals)
\end{itemize}


\textbf{Metrics \& plots:}

\begin{itemize}
\item \textbf{Authoritative metrics}: data/gaia/outputs/mw\_rar\_starlevel\_full\_metrics.txt (global + per-bin residuals)
\item \textbf{All-model summary}: data/gaia/outputs/mw\_all\_model\_summary.png (8-panel RAR + histograms)
\item \textbf{Improved RAR comparison}: data/gaia/outputs/mw\_rar\_comparison\_full\_improved.png (smoothed Σ curve + residual metrics)
\item \textbf{Radial residual map}: data/gaia/outputs/mw\_radial\_residual\_map.png (smooth transition proof)
\item \textbf{Δ histograms}: data/gaia/outputs/mw\_delta\_histograms.png (bias distributions)
\item \textbf{Radial-bin table}: data/gaia/outputs/mw\_radial\_bin\_table.png (per-bin improvement factors)
\item \textbf{Outer rotation curves}: data/gaia/outputs/mw\_outer\_rotation\_curves.png (6–25 kpc v\_circ comparison)
\end{itemize}


\textbf{Analysis documentation:}

\begin{itemize}
\item \textbf{Inner-disk integration analysis}: data/gaia/outputs/INNER\_DISK\_INTEGRATION\_ANALYSIS.md (197 lines; sampling artifact resolution)
\item \textbf{Improved comparison README}: data/gaia/outputs/IMPROVED\_COMPARISON\_README.md (177 lines; smoothed curve methodology)
\item \textbf{Academic plots guide}: data/gaia/outputs/ACADEMIC\_PLOTS\_GUIDE.md (314 lines; objection rebuttals + figure captions)
\end{itemize}


\textbf{Commands to reproduce:}

```bash

\# 1. Fetch inner-disk stars with RVs (Gaia DR3)

python scripts/fetch\_gaia\_wedges.py \

  --max\_stars 20000 --abs\_l\_max 30 --abs\_b\_max 10 \

  --r\_min\_kpc 2 --r\_max\_kpc 6 --require\_rv \

  --out data/gaia/gaia\_inner\_rvs\_20k.csv


\# 2. Convert to NPZ and merge with extended dataset

python scripts/convert\_gaia\_csv\_to\_npz.py \

  --csv data/gaia/gaia\_inner\_rvs\_20k.csv \

  --out data/gaia/gaia\_inner\_rvs\_20k.npz


python scripts/merge\_gaia\_datasets.py \

  --base data/gaia/mw\_gaia\_extended.npz \

  --new data/gaia/gaia\_inner\_rvs\_20k.npz \

  --out data/gaia/mw\_gaia\_full\_coverage.npz


\# 3. Predict star speeds (GPU; uses fit\_params.json from vendor pipeline)

python scripts/predict\_gaia\_star\_speeds.py \

  --npz data/gaia/mw\_gaia\_full\_coverage.npz \

  --fit data/gaia/outputs/mw\_pipeline\_run\_vendor/fit\_params.json \

  --out data/gaia/outputs/mw\_gaia\_full\_coverage\_predicted.csv --device 0


\# 4. Generate star-level RAR metrics + comprehensive plots

python scripts/analyze\_mw\_rar\_starlevel.py \

  --pred\_csv data/gaia/outputs/mw\_gaia\_full\_coverage\_predicted.csv \

  --out\_prefix data/gaia/outputs/mw\_rar\_starlevel\_full --hexbin


python scripts/make\_mw\_rar\_comparison.py \

  --pred\_csv data/gaia/outputs/mw\_gaia\_full\_coverage\_predicted.csv \

  --out\_png data/gaia/outputs/mw\_rar\_comparison\_full\_improved.png


python scripts/generate\_radial\_residual\_map.py

python scripts/generate\_delta\_histograms.py

python scripts/generate\_radial\_bin\_table\_plot.py

python scripts/generate\_outer\_rotation\_curves.py

python scripts/generate\_all\_model\_summary.py

```


\textit{Figure C2. Triaxial lever arm for A2261: θ\_E as a function of q\_LOS under the same kernel and baryons.}


Table C1 — Training clusters (N≈10; auto‑generated)

(see tables/table\_c1.md)


\begin{table}[h]
\centering
\begin{tabular}{llllllllll}
\toprule
Name & z\_l & R500 [kpc] & Σ\_baryon source & Geometry priors & P(z\_s) model & θ\_E(obs) [\"] & θ\_E(pred) [\"] & Residual & Z‑score \\
\midrule
(see scripts/generate\_table\_c1.py) \\
\bottomrule
\end{tabular}
\end{table}


Table C2 — Population posteriors (N≈10; NUTS‑grid)

(see tables/table\_c2.md)


\begin{table}[h]
\centering
\begin{tabular}{lll}
\toprule
Parameter & Posterior & Notes \\
\midrule
μ\_A & 4.6 ± 0.4 & population mean amplitude \\
σ\_A & ≈ 1.5 & intrinsic scatter \\
ℓ₀,⋆ & ≈ 200 kpc & reference coherence length \\
γ & 0.09 ± 0.10 & mass‑scaling (consistent with 0) \\
ΔWAIC (γ‑free vs γ=0) & 0.01 ± 2.5 & inconclusive \\
\bottomrule
\end{tabular}
\end{table}


\medskip\hrule\medskip


\section{6. Discussion}


Where Σ‑Gravity stands after §§3–5. The Newtonian/GR limit is recovered locally; a single, conservative kernel (calibrated once per domain) reaches 0.087 dex RAR scatter on SPARC and reproduces cluster Einstein radii using realistic baryons and triaxial geometry. Current data are consistent with no mass‑scaling of ℓ₀ (γ = 0.09 ± 0.10); the safety margin against Solar‑System bounds remains large. We outline limitations and tests that could falsify or sharpen the framework.


\textbf{Mass‑scaling.} After corrections, the posterior for γ peaks near zero with 1σ ≈ 0.10. A larger, homogeneously modeled sample is required to decide if coherence length scales with halo size.


\textbf{Amplitude ratio consistency check (path-counting).} The empirically calibrated amplitude ratio A\_c/A\_0 ≈ 4.6/0.591 ≈ 7.8 between clusters and galaxies is consistent with theoretical expectations from path geometry. If A scales with the number of near-stationary path families, then dimensionality alone suggests a significant enhancement: galaxy rotation curves sample paths primarily confined to a 2-D disk (∼2π radians of azimuthal freedom), whereas cluster lensing integrates over 3-D source volumes with full 4π steradian solid angle and line-of-sight depths of order 2R\_500 ∼ 2 Mpc (compared to disk scale heights h\_z ∼ 1 kpc). A rough combinatorial estimate,


\begin{equation}
\frac{A_c}{A_0} \sim \frac{\Omega_{\mathrm{cluster}}}{\Omega_{\mathrm{gal}}} \times \frac{L_{\mathrm{cluster}}}{L_{\mathrm{disk}}} \sim \frac{4\pi}{2\pi} \times \frac{1000\,\mathrm{kpc}}{20\,\mathrm{kpc}} \sim 100,
\end{equation}


is an overestimate (not all paths contribute equally; coherence weighting and geometry factors reduce this). A more careful calculation accounting for the elliptic-integral ring geometry (Appendix B) and exponential coherence decay yields a ratio of O(10), in quantitative agreement with the observed ≈7.8. This supports the interpretation of A as a measure of coherent path-family count rather than a disconnected fit parameter. Variations with cluster triaxiality (oblate vs prolate; q\_LOS ∈ [0.7, 1.3]) and galaxy disk thickness offer direct tests; triaxial sensitivity of ∼20–30\% in θ\_E is already confirmed (§5.3, Figure C2).


\textbf{Major open items and how we address them.}

\begin{enumerate}
\item \textbf{Sample bias \& redshift systematics} → explicit D\_LS/D\_S, cluster‑specific M\_500, triaxial posteriors, and measured P(z\_s); expanding to N≈18 Tier‑1+2 clusters.
\item \textbf{Outliers \& mergers} → multi‑component Σ or temperature/entropy gates for shocked ICM; test with weak‑lensing profiles and arc redshifts.
\item \textbf{Physical origin of A\_c, ℓ\_0, and γ} → stationary‑phase kernel in progress; γ is \textbf{falsifiable}.
\item \textbf{Model comparison} → γ‑free vs γ=0 with ΔBIC/WAIC; blind PPC on hold‑outs.
\end{enumerate}


\medskip\hrule\medskip


\section{7. Predictions \& falsifiability}


• Triaxial lever arm: θ\_E should change by ≈15–30\% as q\_LOS varies from ~0.8 to 1.3.  

• Weak lensing: Σ‑Gravity predicts shallower γ\_t(R) at 100–300 kpc than Newton‑baryons; stacking N≳18 clusters should distinguish.  

• Mergers: shocked ICM decoheres; lensing tracks unshocked gas + BCG → offset prediction.  

• Solar System / binaries: no detectable anomaly; PN bounds ≪10^−5.


\medskip\hrule\medskip


\section{8. Cosmological Implications and the CMB}


While a full cosmological treatment is deferred, Scale‑Dependent Quantum Coherence provides a natural, testable narrative for the CMB and late‑time structure.


\subsection{8.1. A state‑dependent coherence length}


We propose that $\tau_{\rm collapse}$ (and thus $\ell_0=c\,\tau_{\rm collapse}$) depends on the physical state of baryons: hot, dense, rapidly interacting plasmas act as efficient “measuring devices,” shortening coherence; cold, ordered, low‑entropy media preserve it.


\subsection{8.2. Early universe and acoustic peaks}


Prior to recombination ($t<380{,}000$ yr), the tightly coupled photon‑baryon plasma continually measures spacetime, rendering $\ell_0$ microscopic. On cosmological scales the universe is vastly larger than $\ell_0$, so gravity behaves classically and the acoustic peak locations match ΛCDM. The standard sound horizon ruler is preserved.


\subsection{8.3. A gravitational phase transition at recombination}


At recombination, photon scattering shuts off. We hypothesize a rapid increase in $\tau_{\rm collapse}\Rightarrow\ell_0$, initiating macroscopic gravitational coherence in bound systems. If non‑instantaneous, this can subtly modulate peak heights at last scattering—a small, distinctive signature for next‑generation CMB data.


\subsection{8.4. Late‑time ISW and structure formation}


In Σ‑Gravity the potential is $\Phi_{\rm bar}[1+\mathcal{K}(t,\mathbf{x})]$. As structures cross the $\ell_0$ threshold, $\mathcal{K}$ turns on, non‑linearly deepening wells. This yields a distinct late‑time ISW cross‑correlation between CMB temperature and large‑scale structure compared to ΛCDM. Existing ISW anomalies may be naturally accommodated.


\subsection{8.5. Future Directions and Cosmological Frontiers}


\subsubsection{Hypothesis (speculative): evolving coherence and effective redshift}


The following ideas are exploratory and not used in §§3–5. Cosmological redshift could arise from a slowly relaxing quantum vacuum: an initially high‑coherence state (large $\mathcal{K}$) relaxes toward $\mathcal{K}\to0$, lifting the baseline gravitational potential. Photons might lose energy by climbing this rising floor, producing redshift; time dilation would follow as $(1+z)$ from gravitational time dilation in the deeper past potential. Each claim should be tested per §8.5 (e.g., fit a minimal $\mathcal{K}(t)$ to SNe/BAO; AP test; CMB–LSS cross‑correlation).


\subsubsection{Falsifiable cosmological tests}


\begin{itemize}
\item Redshift–distance: Fit a minimal, physically motivated decay law $\mathcal{K}(t)$ to SNe and BAO; test parsimony vs ΛCDM.
\item Alcock–Paczyński: In a non‑expanding metric, statistically spherical objects/correlations remain isotropic—absence of ΛCDM’s geometric distortion is decisive.
\item CMB/ISW: Predict a unique CMB–LSS cross‑correlation from evolving $\mathcal{K}$; distinguishable from ΛCDM.
\item Bullet Cluster: Shock fronts act as “measurements,” forcing local $\mathcal{K}\to0$. Lensing should follow BCG + unshocked gas, explaining the offset without particles.
\end{itemize}


\subsubsection{Theoretical roadmap}


Derive a first‑principles decoherence law $\mathcal{K}(t)$ to fix the redshift–distance relation a priori; extend linear perturbations and weak‑lensing kernels $K(k)$; confront Planck lensing and shear two‑point data.


\medskip\hrule\medskip


\section{9. Reproducibility \& code availability}


\subsection{9.0 Milky Way (Gaia DR3) — exact replication (this repo)}


\begin{enumerate}
\item Fit MW pipeline (GPU; writes fit\_params.json)
\end{enumerate}

```pwsh path=null start=null

python -m vendor.maxdepth\_gaia.run\_pipeline --use\_source mw\_csv --mw\_csv\_path "data/gaia/mw/gaia\_mw\_real.csv" --saveplot "data/gaia/outputs/mw\_pipeline\_run\_vendor/mw\_rotation\_curve\_maxdepth.png"

```


\begin{enumerate}
\item Predict star‑level speeds (GPU)
\end{enumerate}

```pwsh path=null start=null

python scripts/predict\_gaia\_star\_speeds.py --npz "data/gaia/mw/mw\_gaia\_144k.npz" --fit "data/gaia/outputs/mw\_pipeline\_run\_vendor/fit\_params.json" --out "data/gaia/outputs/mw\_gaia\_144k\_predicted.csv" --device 0

```


\begin{enumerate}
\item Star‑level RAR table, metrics, and plot
\end{enumerate}

```pwsh path=null start=null

python scripts/analyze\_mw\_rar\_starlevel.py --pred\_csv "data/gaia/outputs/mw\_gaia\_144k\_predicted.csv" --out\_prefix "data/gaia/outputs/mw\_rar\_starlevel" --hexbin

```


\begin{enumerate}
\item Comparison plot (R vs g medians; RAR line‑fits Σ vs MOND vs NFW vs GR)
\end{enumerate}

```pwsh path=null start=null

python scripts/make\_mw\_rar\_comparison.py

```


Notes

\begin{itemize}
\item Requires CuPy/NVIDIA GPU for steps 1–2; steps 3–4 are CPU.
\item All input data are under data/gaia; outputs are written under data/gaia/outputs.
\item For MOND/NFW baselines, parameters are read from fit\_params.json (a0, V200, c).
\end{itemize}


\subsection{9.1. Repository structure \& prerequisites}


Python ≥3.10; NumPy/SciPy/Matplotlib; pymc≥5; optional: emcee, CuPy (GPU), arviz.


\subsection{9.2. Galaxy (RAR) pipeline}


\begin{enumerate}
\item Validation:
\end{enumerate}

python many\_path\_model/validation\_suite.py --all  

Produces VALIDATION\_REPORT.md and btfr\_rar\_validation.png.


\begin{enumerate}
\item Optimization:
\end{enumerate}

python many\_path\_model/run\_full\_tuning\_pipeline.py  

Outputs best\_hyperparameters.json, ablation\_results.json, holdout\_results.json.


\begin{enumerate}
\item Key file: many\_path\_model/path\_spectrum\_kernel.py (stationary‑phase path spectrum kernel).
\end{enumerate}


\subsection{9.3. Cluster (Σ‑kernel) pipeline}


\begin{enumerate}
\item Baryons:
\end{enumerate}

core/gnfw\_gas\_profiles.py, core/build\_cluster\_baryons.py (f\_gas=0.11, clumping fix), data/clusters/*.json; per‑cluster Σ\_baryon(R) CSVs ingested when available (A2261, MACSJ1149 hold‑outs).


\begin{enumerate}
\item Triaxial projection:
\end{enumerate}

core/triaxial\_lensing.py (global normalization; geometry validated in docs/triaxial\_lensing\_fix\_report.md).


\begin{enumerate}
\item Projected kernel:
\end{enumerate}

core/kernel2d\_sigma.py (K\_Σ(R)=A\_c·C(R;ℓ₀,⋯)).


\begin{enumerate}
\item Diagnostics (MACS0416):
\end{enumerate}

python scripts/plot\_macs0416\_diagnostics.py  

Generates: convergence\_profiles.png, cumulative\_mass.png, convergence\_maps\_2d.png, boost\_profile.png.


\subsection{9.4. Triaxial tests \& Einstein mass checks}


python scripts/simple\_einstein\_check.py  

python scripts/test\_macs0416\_triaxial\_kernel.py  

Outputs geometry sensitivity figs and θ\_E validation.


\subsection{9.5. Hierarchical calibration}


• Tier‑1 clean (5 relaxed clusters):  

python scripts/run\_hierarchical\_tier12\_clean.py → μ\_A, σ\_A, χ²/d.o.f.  

• MCMC (fast geometry model):  

python scripts/run\_tier12\_mcmc\_fast.py → posterior\_A\_c.png, summary.txt


\subsection{9.6. Blind hold‑outs (with overrides)}


\subsection{9.7. Lensing visuals (κ and α) — quick reproduction (this repo)}


Self-contained figures for convergence and deflection, calibrated to the observed θ\_E in the catalog and using the paper’s Σ‑kernel K(R)=A\_c·C(R;ℓ₀,⋯):


```pwsh

python scripts/make\_cluster\_lensing\_profiles.py --clusters "MACS1149" --fb 0.33 --ell0\_frac 0.60 --p 2 --ncoh 2

```


\begin{itemize}
\item Input: data/clusters/master\_catalog.csv (uses cluster\_name, theta\_E\_obs\_arcsec)
\item Output: data/clusters/figures/<name>\_kappa\_profiles.png and <name>\_alpha\_profiles.png
\item Notes: This is a didactic, axisymmetric visual (SIS toys for GR(baryons) and GR+DM). For baryon‑accurate, triaxial panels, use the full cluster pipeline scripts and per‑cluster Σ\_baryon(R).
\end{itemize}


Build multi‑cluster panels for the paper:


```pwsh

python scripts/make\_cluster\_lensing\_panels.py

```


Produces: figures/cluster\_kappa\_profiles\_panel.png and figures/cluster\_alpha\_profiles\_panel.png.


python scripts/run\_holdout\_validation.py → pred\_vs\_obs\_holdout.png  

python scripts/validate\_holdout\_mass\_scaled.py \

  --posterior output/n10\_nutsgrid/flat\_samples.npz \

  --catalog data/clusters/master\_catalog.csv \

  --pzs median --check-training 1 \

  --overrides-dir data/overrides


Artifacts are stored under output/… and results/…; each run writes a manifest (catalog MD5, overrides JSON, kernel mode, Σ\_baryon source, P(z\_s), sampler, seed).


\medskip\hrule\medskip


\section{10. What changed since the last draft}


• Fixed Newtonian‑limit, unit, and clumping‑sign bugs; unified f\_gas normalization.  

• Replaced spherical 3‑D shell kernel by projected 2‑D Σ‑kernel to preserve triaxial geometry; restored ~60\% Σ‑sensitivity and ~20–30\% θ\_E lever arm.  

• Switched to differentiable θ\_E surrogate + PyMC NUTS; ΔWAIC ≈ 0 ± 2.5 for γ‑free vs γ=0.  

• Curated N=10 training set with per‑cluster Σ(R) and P(z\_s) mixtures; blind hold‑outs A2261 + MACSJ1149 both inside 68\% PPC; median fractional error 14.9\%.


\medskip\hrule\medskip


\section{11. Planned analyses \& roadmap}


Immediate (clusters): expand to N≈18; test γ via ΔBIC; stack γ\_t(R).


Galaxies: finalize v1.0 RAR release (archive hyperparameters, seeds, splits, plots).


Cross‑checks: BTFR residuals vs morphology; cluster gas systematics; BCG/ICL M/L tests; mocks.


\subsection{11.1. State of the union (Solar → Galaxy → Cluster)}


\begin{itemize}
\item Solar System — Pass: Kernel gates collapse locally (K→0); PPN/Cassini‑safe.
\item Disk galaxies — Strong: SPARC RAR ≈0.087 dex; BTFR/RC cross‑checks pass.
\item Clusters — Population: μ\_A≈4.6, σ\_A≈1.5; γ consistent with 0.
\end{itemize}


\medskip\hrule\medskip


\section{12a. Figures (paper bundle)}


\begin{enumerate}
\item Galaxies — RAR (SPARC‑166): figures/rar\_sparc\_validation.png
\item Galaxies — BTFR (two‑panel): figures/btfr\_two\_panel\_v2.png
\item Galaxies — RC gallery (12‑panel): figures/rc\_gallery.png
\item Galaxies — RC residual histogram: figures/rc\_residual\_hist.png
\item Clusters — Hold‑outs predicted vs observed: figures/holdouts\_pred\_vs\_obs.png
\item Clusters — K‑fold predicted vs observed: figures/kfold\_pred\_vs\_obs.png
\item Clusters — K‑fold coverage (68\%): figures/kfold\_coverage.png
\item Clusters — ⟨κ⟩(<R) panels: figures/cluster\_kappa\_panels.png
\item Clusters — Triaxial sensitivity: figures/triaxial\_sensitivity\_A2261.png
\item Methods — MACS0416 convergence profiles: figures/macs0416\_convergence\_profiles.png
\item Clusters — Convergence panels (all): figures/cluster\_kappa\_profiles\_panel.png
\item Clusters — Deflection panels (all): figures/cluster\_alpha\_profiles\_panel.png
\end{enumerate}


\section{13. Conclusion}


Σ‑Gravity implements a coherence‑gated, multiplicative kernel that preserves GR locally and explains galaxy and cluster phenomenology with realistic baryons. With no per‑galaxy tuning, the model matches the SPARC RAR at 0.087 dex, and with triaxial projection and Σ\_crit integrates cluster lensing to achieve μ\_A ≈ 4.6 and blind hold‑out success at the 68\% level. The open question is whether ℓ₀ scales with halo size; present constraints favor γ≈0. The next steps are larger homogeneous cluster samples (with P(z\_s)) and stacked weak‑lensing profiles to test the predicted geometric lever arm.


\medskip\hrule\medskip


\section{Acknowledgments}


We thank collaborators and the maintainers of the SPARC database and strong‑lensing compilations. Computing performed with open‑source Python tools.


\medskip\hrule\medskip


\section{Data \& code availability}


All scripts listed in §9 are included in the project repository; outputs (CSV/JSON/PNG) are generated deterministically from checked‑in configs.


\medskip\hrule\medskip


\section{Appendix A — Integration‑by‑parts and cancellation of O(v/c)}


We outline a weak‑field, post‑Newtonian (PN) expansion consistent with causality. Using mass continuity $\dot\rho=-\nabla'\!\cdot(\rho\,\mathbf{v})$ and periodic/axisymmetric boundaries, the linear $\mathcal{O}(v/c)$ term vanishes after integration by parts, leaving the leading correction at $\mathcal{O}(v^2/c^2)$. For illustration we write the Poisson‑limit potential kernel $1/\lvert \mathbf{x}-\mathbf{x}'\rvert$; this is a PN convenience, not a full GR Green’s‑function solution:


\begin{equation}
\delta\Phi(\mathbf{x}) = \frac{G}{2c^2} \int \frac{\nabla'\!\cdot(\rho\,\mathbf{v}\!\otimes\!\mathbf{v})}{\lvert \mathbf{x}-\mathbf{x}'\rvert}\,\mathrm{d}^3\!x' ,\qquad
\delta\mathbf{g}(\mathbf{x}) = -\frac{G}{2c^2} \int \nabla\!\left(\frac{1}{\lvert \mathbf{x}-\mathbf{x}'\rvert}\right) \, \nabla'\!\cdot(\rho\,\mathbf{v}\!\otimes\!\mathbf{v})\,\mathrm{d}^3\!x' .
\end{equation}


Example (circular flow): for $\mathbf{v}=v_\phi\,\hat\phi$ in an axisymmetric disk, only the divergence of the Reynolds‑stress‑like tensor contributes; the induced field is curl‑free by construction.


\section{Appendix B — Elliptic ring kernel (exact geometry)}


The azimuthal integral reduces to complete elliptic integrals with dimensionless parameter

\[

 m \;\equiv\; \frac{4 R R'}{(R+R')^2} \in [0,1].

\]

Then

\[

\int\_{0}^{2\pi} \frac{d\varphi}{\sqrt{R^2 + R'^2 - 2 R R'\cos\varphi}} \;=\; \frac{4}{R+R'}\,K(m).

\]


Reference check (relative error < 1e−6):


```python

import numpy as np

from mpmath import quad, ellipk


def ring\_green\_numeric(R, Rp):

    f = lambda phi: 1.0/np.sqrt(R\textbf{2 + Rp}2 - 2\textit{R}Rp*np.cos(phi))

    return 2.0 * quad(f, [0, np.pi])


def ring\_green\_elliptic(R, Rp):

    m = 4.0\textit{R}Rp/((R+Rp)**2)  \# dimensionless parameter m \in [0,1]

    return 4.0/(R+Rp) * ellipk(m)


R, Rp = 5.0, 7.0

num = ring\_green\_numeric(R, Rp)

ana = ring\_green\_elliptic(R, Rp)

assert abs(num-ana)/num < 1e-6

```


\section{Appendix C — Stationary phase \& coherence window}


Near the stationary azimuth $\varphi=0$ one may expand the separation as $\Delta(\varphi)\approx D + (RR'/(2D))\,\varphi^2$. The phase integral reduces to a Gaussian/Fresnel form; adding stochastic dephasing over a coherence length $\ell_0$ yields a radial envelope equivalent to


\begin{equation}
C(R) = 1 - \Big[1 + (R/\ell_0)^p\Big]^{-n_{\rm coh}} ,
\end{equation}


with phenomenological exponents $p,n_{\rm coh}$ calibrated once on data. This envelope multiplies the Newtonian response, remaining curl‑free.


\subsection{C.1 Superstatistical derivation of the coherence window}


We now show that the functional form of C(R) is not arbitrary but emerges naturally from a stochastic decoherence model in a heterogeneous medium. This derivation applies a standard mixture identity from reliability theory (Gamma–Weibull compounding yields Burr-XII survival; see, e.g., MATLAB Statistics Toolbox documentation and Rodriguez 1977) to the novel context of gravitational decoherence channels.



\begin{equation}
S(R|\lambda) = e^{-\lambda(R/\ell_0)^p}.
\end{equation}


To represent this heterogeneity, we model λ as drawn from a Gamma distribution with shape n\_coh and rate β:


\begin{equation}
\lambda \sim \mathrm{Gamma}(n_{\mathrm{coh}}, \beta).
\end{equation}


\textbf{Derivation.} The marginal survival probability is the expectation over λ:


\begin{equation}
S(R) = \mathbb{E}_{\lambda}\left[e^{-\lambda(R/\ell_0)^p}\right] = \int_0^\infty e^{-\lambda(R/\ell_0)^p} \frac{\beta^{n_{\mathrm{coh}}}}{\Gamma(n_{\mathrm{coh}})} \lambda^{n_{\mathrm{coh}}-1} e^{-\beta\lambda} d\lambda.
\end{equation}


Using the Laplace transform of the Gamma distribution,


\begin{equation}
S(R) = \left(\frac{\beta}{\beta + (R/\ell_0)^p}\right)^{n_{\mathrm{coh}}} = \left[1 + \frac{(R/\ell_0)^p}{\beta}\right]^{-n_{\mathrm{coh}}}.
\end{equation}


Absorbing the rate parameter β into the definition of ℓ₀ (ℓ₀′ ≡ ℓ₀ β^(1/p)), the coherence window is


\begin{equation}
C(R) = 1 - S(R) = 1 - \left[1 + \left(\frac{R}{\ell_0}\right)^p\right]^{-n_{\mathrm{coh}}}.
\end{equation}


This is the Burr Type XII (Singh–Maddala) cumulative distribution function.


\textbf{Interpretation.} The fitted parameters now have direct physical meaning:  

\begin{itemize}
\item \textbf{ℓ₀}: characteristic coherence scale set by local decoherence timescale τ\_collapse
\item \textbf{p}: encodes how interactions accumulate with scale (p < 1 suggests correlated/sparse interactions; p = 2 would be area-like)
\item \textbf{n\_coh}: effective number of independent decoherence channels; larger n\_coh implies narrower variability in λ (more homogeneous environment)
\end{itemize}


\textbf{Testable predictions.} If this interpretation is correct, n\_coh should increase in relaxed, homogeneous systems (ellipticals, relaxed clusters) and decrease in turbulent, clumpy environments (barred galaxies, merger clusters). The exponent p should shift systematically with morphology. Splitting galaxies by bar fraction or clusters by entropy/merger stage offers direct empirical tests (Bridge 1 in §6).


\textbf{Attribution.} The Gamma–Weibull → Burr-XII identity is standard (Rodriguez 1977, JSTOR; MATLAB docs); our contribution is the application to gravitational decoherence and the physical interpretation of {ℓ₀, p, n\_coh} in terms of path coherence and environmental heterogeneity. For the broader "superstatistics" framework (heterogeneous rate parameters), see Beck \& Cohen 2003, arXiv:cond-mat/0303288.


\section{Appendix D — PN error budget}


We bound neglected terms by


\begin{equation}
\Delta_{\rm PN} \;\lesssim\; C_1\,(v/c)^3 \, + \, C_2\,(v/c)^2\,(H/R) \, + \, C_3\,(v/c)^2\,(R/R_\Sigma)\,.
\end{equation}


In disks and clusters, representative values place all terms $\ll10^{-5}$, well below statistical errors. (See PN bounds figure for a SPARC galaxy.)


\section{Appendix E — Data, code, and reproducibility (one‑stop)}


Environment: Python ≥3.10; numpy/scipy/pandas/matplotlib; pymc≥5; optional emcee, CuPy, arviz.


Exact commands (galaxies/clusters; matches §9): see scripts listed there. A convenience runner scripts/make\_paper\_figures.py executes the full figure pipeline and writes MANIFEST.json with catalog MD5, seed, timestamps, and produced artifact paths.


Provenance: each run writes a manifest (catalog MD5, overrides JSON, kernel mode, P(z\_s), seed, sampler diagnostics). Expected outputs include: RAR = 0.087 dex; 5‑fold RAR = 0.083±0.003; cluster hold‑outs coverage 2/2 with 14.9\% median fractional error.


Regression tests: Solar‑System/PPN and wide‑binary safety; legacy galaxy runs still pass under the updated kernel gates.


\medskip\hrule\medskip


\medskip\hrule\medskip


\medskip\hrule\medskip


\section{Appendix F — Σ‑Gravity rigorous derivation (PRD excerpt)}


This appendix incorporates the core theoretical derivation and proofs from the companion PRD‑style manuscript, serving as backup for the kernel form, curl‑free structure, Solar‑System safety, and amplitude scaling between galaxies and clusters.


\section{I. FUNDAMENTAL POSTULATES}


\subsection{A. Gravitational Field as Quantum Superposition}


\textbf{Postulate I}: In the absence of strong decoherence, the gravitational field exists as a superposition of geometric configurations characterized by different path histories.


Mathematically, for a test particle moving from point A to B, the propagator is:


```

K(B,A) = ∫ D[path] exp(iS[path]/ℏ)     (1)

```


where S[path] is the action along each geometric path.


\textbf{Justification}: This is standard path‑integral quantum mechanics, applied to gravity. The novelty is in recognizing that decoherence rates differ dramatically between compact and extended systems.


\subsection{B. Scale‑Dependent Decoherence}


\textbf{Postulate II}: Geometric superpositions collapse to classical configurations on a characteristic timescale τ\_collapse(R) that depends on the spatial scale R and matter density ρ.


\textbf{Physical Mechanism}: We propose that gravitational geometries decohere through continuous weak measurement by matter. Unlike quantum systems that decohere via environmental entanglement (photon scattering, etc.), gravity decoheres through \textbf{self‑interaction} with the mass distribution that sources it.


The decoherence rate is proportional to the rate at which matter "samples" different geometric configurations:


```

Γ\_decoherence(R) ~ (interaction rate) × (geometric variation)     (2)

```


For a region of size R with density ρ:

\begin{itemize}
\item Interaction rate ~ ρ (more mass → more interactions)
\item Geometric variation ~ R² (larger regions have more distinct paths)
\end{itemize}


Therefore:

```

τ\_collapse(R) ~ 1/(ρ G R² α)     (3)

```


where α is a dimensionless constant of order unity characterizing the efficiency of gravitational self‑measurement.


\textbf{Key Insight}: This gives a coherence length scale:

```

ℓ\_0 = √(c/(ρ G α))     (4)

```


For typical galaxy halo densities ρ ~ 10⁻²¹ kg/m³:

```

ℓ\_0 ~ √(3×10⁸ / (10⁻²¹ × 6.67×10⁻¹¹ × 1)) ~ 7×10¹⁹ m ~ 2 kpc     (5)

```


Order of magnitude correct; ℓ\_0 naturally lands at galactic scales.


\medskip\hrule\medskip


\section{II. DERIVATION OF THE ENHANCEMENT KERNEL}


\subsection{A. Weak‑Field Expansion}


```

```


Newtonian potential:

```

```


\subsection{B. Path Sum and Stationary Phase}


```

```


Stationary phase:

```

S[path] = S\_classical + (1/2)δ²S[deviation] + ...     (9)

```


gives a near‑classical amplitude factor:

```

∫ D[path] exp(iS/ℏ) ≈ A\_0 exp(iS\_classical/ℏ) [1 + quantum corrections]     (10)

```


\subsection{C. Coherence Weighting}


Probability that a path of extent R remains coherent:

```

P\_coherent(R) = exp(-∫ dt/τ\_collapse(r(t)))     (11)

```


For characteristic scale R in density ρ:

```

P\_coherent(R) ≈ exp(-(R/ℓ\_0)^p)     (12)

```


with p ≈ 2. A smooth, causal window:

```

C(R) = 1 - [1 + (R/ℓ\_0)^p]^(-n\_coh)     (13)

```


\subsection{D. Multiplicative Structure}


Classical contribution from dV:

```

```

Quantum‑enhanced contribution:

```

```

with

```

[coherent path sum] ≈ [1 + A · C(R)]     (16–17)

```

Hence

```

Φ\_eff = Φ\_classical [1 + K(R)],

 g\_eff ≈ g\_classical [1 + K(R)]     (18–20)

```


\medskip\hrule\medskip


\section{III. CURL‑FREE PROPERTY}


For axisymmetric systems with K=K(R):

```

∇ × g\_eff = (∇ × g\_bar)(1+K) + ∇K × g\_bar = 0     (21–22)

```

so the enhanced field remains conservative.


\medskip\hrule\medskip


\section{IV. SOLAR SYSTEM CONSTRAINTS}


\begin{equation}
K(1\,\mathrm{AU}) \sim 10^{-7} \ll 10^{-5}
\end{equation}

Safety margin ≥100×.


\medskip\hrule\medskip


\section{V. AMPLITUDE SCALING: GALAXIES VS CLUSTERS}


Path‑counting (2D disks vs 3D clusters) predicts A\_cluster/A\_gal ~ O(10). Empirically ≈7.7; consistent to order‑unity after geometry factors.


\medskip\hrule\medskip


\section{VI. QUANTITATIVE PREDICTIONS}


\begin{itemize}
\item Galaxies (SPARC): RAR scatter ≈0.087 dex; BTFR ≈0.15 dex (using A\_gal≈0.6, ℓ\_0≈5 kpc).
\item Clusters: θ\_E accuracy ≈15\% with A\_cluster≈4.6; triaxial lever arm 20–30\%.
\item Solar System: K(0.1–1000 AU) < 10⁻⁶.
\end{itemize}


\medskip\hrule\medskip


\section{F. Technical addenda (selected)}


\subsection{F.1 Dimensional analysis of ℓ\_0}


We treat ℓ₀ as a coherence length defined by a collapse time: ℓ₀ ≡ c\,τ\_{\rm collapse}. A common heuristic scales the collapse rate with the gravitational interaction rate, τ\_{\rm collapse}^{-1} ∝ \sqrt{\alpha\,\rho G}, which yields the length‑scale estimate

\begin{equation}
\ell_0 \sim \frac{c}{\sqrt{\alpha\,\rho G}}.
\end{equation}

This is a scaling ansatz (not a derivation); ℓ₀ is calibrated empirically in §§5.1–5.3.


\subsection{F.2 Numerical kernel (example)}


```python

def sigma\_gravity\_kernel(R\_kpc, A=0.6, ell\_0=5.0, p=0.75, n\_coh=0.5):

    C = 1 - (1 + (R\_kpc/ell\_0)\textbf{p)}(-n\_coh)

    return A * C

```


\subsection{F.3 Ring kernel expression}


\begin{equation}
G_{\mathrm{ring}}(R, R') = \int_{0}^{2\pi} \frac{d\varphi}{\sqrt{R^2 + R'^2 - 2 R R'\cos\varphi}}
\end{equation}


Tables (Appendix F):

\begin{itemize}
\item Table F1 — Galaxy parameter sensitivity (ablation \& sweeps): \texttt{many\_path\_model/paper\_release/tables/galaxy\_param\_sensitivity.md}
\item Table F2 — Cluster parameter sensitivity (MACS0416): \texttt{many\_path\_model/paper\_release/tables/cluster\_param\_sensitivity.md}
\item Table F3 — Cluster sensitivity across N≈10 (Tier 1/2): \texttt{many\_path\_model/paper\_release/tables/cluster\_param\_sensitivity\_n10.md}
\end{itemize}


For full derivations and proofs, see the PRD manuscript draft archived with this paper.



\end{document}
