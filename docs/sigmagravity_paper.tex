\documentclass[11pt,a4paper]{article}

% Packages
\usepackage[utf8]{inputenc}
\usepackage[T1]{fontenc}
\usepackage{amsmath,amssymb,amsfonts}
\usepackage{graphicx}
\usepackage[margin=20mm]{geometry}
\usepackage{hyperref}
\usepackage{booktabs}
\usepackage{xcolor}
\usepackage{longtable}
\usepackage{ragged2e}  % Better paragraph formatting

% Line breaking settings
\sloppy  % Allow more flexible line breaking
\hyphenpenalty=1000  % Reduce hyphenation
\exhyphenpenalty=1000  % Reduce hyphenation in compound words
\doublehyphendemerits=10000  % Penalize consecutive hyphenated lines

% Math line breaking
\allowdisplaybreaks  % Allow page breaks in display math
\binoppenalty=1000   % Penalty for breaking after binary operators
\relpenalty=1000     % Penalty for breaking after relations

% Unicode support for special characters
\DeclareUnicodeCharacter{03A3}{$\Sigma$}  % Σ
\DeclareUnicodeCharacter{2020}{$\dagger$} % †
\DeclareUnicodeCharacter{2113}{$\ell$}    % ℓ
\DeclareUnicodeCharacter{2192}{$\to$}     % →
\DeclareUnicodeCharacter{2261}{$\equiv$}  % ≡
\DeclareUnicodeCharacter{00D7}{$\times$}  % ×
\DeclareUnicodeCharacter{2211}{$\sum$}    % ∑
\DeclareUnicodeCharacter{220F}{$\prod$}   % ∏
\DeclareUnicodeCharacter{2032}{$'$}       % ′
\DeclareUnicodeCharacter{2264}{$\leq$}    % ≤
\DeclareUnicodeCharacter{2265}{$\geq$}    % ≥
\DeclareUnicodeCharacter{226B}{$\gg$}     % ≫
\DeclareUnicodeCharacter{226A}{$\ll$}     % ≪
\DeclareUnicodeCharacter{27E8}{$\langle$} % ⟨
\DeclareUnicodeCharacter{27E9}{$\rangle$} % ⟩
\DeclareUnicodeCharacter{2013}{--}        % en-dash
\DeclareUnicodeCharacter{2014}{---}       % em-dash
\DeclareUnicodeCharacter{2212}{$-$}       % minus
\DeclareUnicodeCharacter{2260}{$\neq$}    % ≠
\DeclareUnicodeCharacter{2248}{$\approx$} % ≈
\DeclareUnicodeCharacter{00B1}{$\pm$}     % ±

% Hyperref setup
\hypersetup{
    colorlinks=true,
    linkcolor=blue,
    citecolor=blue,
    urlcolor=blue,
}

% Custom commands for better line breaking
\newcommand{\allowbreakmath}[1]{$#1$\allowbreak}
\newcommand{\breakablemath}[1]{$#1$\allowbreak}

% Custom commands
\newcommand{\Sigmagrav}{$\Sigma$-Gravity}
\newcommand{\ellzero}{\ell_0}

\begin{document}

\title{Σ-Gravity: Coherent Gravitational Enhancement from Torsion Mode Superposition}

\author{Leonard Speiser}

\date{}

\maketitle

\textbf{Author:} Leonard Speiser  

\textbf{Date:} November 30, 2025


\medskip\hrule\medskip


\begin{abstract}
The observed dynamics of galaxies and galaxy clusters systematically exceed predictions from visible matter alone—a discrepancy conventionally attributed to dark matter. Here we present Σ-Gravity ("Sigma-Gravity"), a framework where \textbf{coherent superposition of gravitational torsion modes} produces scale-dependent enhancement in extended, dynamically cold systems. Built on teleparallel gravity—the mathematical equivalent of General Relativity where gravity is mediated by torsion rather than curvature—the key insight is that torsion contributions from spatially separated mass elements can interfere constructively when their phases remain aligned, analogous to coherent light in a laser or Cooper pairs in a superconductor. The enhancement follows a universal formula Σ = 1 + A × W(r) × h(g), where h(g) = √(g†/g) × g†/(g†+g) encodes acceleration dependence, W(r) encodes spatial coherence decay, and the critical acceleration g† = cH₀/(2e) ≈ 1.2 × 10⁻¹⁰ m/s² emerges from cosmological horizon physics. Applied to 171 SPARC galaxies, Σ-Gravity achieves 0.100 dex mean RAR scatter—matching MOND—while winning head-to-head on 97 vs 74 galaxies. Zero-shot application to the Milky Way rotation curve using McGaugh's baryonic model achieves RMS = 5.7 km/s vs McGaugh/GRAVITY data (Δ = −5.7 km/s at the solar circle). Blind hold-out validation on galaxy clusters achieves 2/2 coverage within 68\% posterior intervals. The theory passes Solar System constraints by 8 orders of magnitude due to automatic coherence suppression in compact systems. Unlike particle dark matter, no per-system halo fitting is required; unlike MOND, Σ-Gravity is embedded in relativistic field theory with g† derived from cosmological constants rather than fitted. The "Σ" refers both to the enhancement factor (Σ ≥ 1) and to the coherent summation of torsion modes that produces it. 
\end{abstract}

\section{Introduction}


\subsection{The Missing Mass Problem}


A fundamental tension pervades modern astrophysics: the gravitational dynamics of galaxies and galaxy clusters systematically exceed predictions from visible matter alone. In spiral galaxies, stars orbit at velocities that remain approximately constant well beyond the optical disk, where Newtonian gravity predicts Keplerian decline. In galaxy clusters, both dynamical masses inferred from galaxy velocities and lensing masses from gravitational light deflection exceed visible baryonic mass by factors of 5-10. This "missing mass" problem has persisted for nearly a century since Zwicky's original cluster observations.


The standard cosmological model (ΛCDM) addresses this through cold dark matter—a hypothetical particle species comprising approximately 27\% of cosmic energy density. Dark matter successfully explains large-scale structure formation and cosmic microwave background anisotropies. However, despite decades of direct detection experiments, no dark matter particle has been identified. The parameter freedom inherent in fitting individual dark matter halos to each galaxy (2-3 parameters per system) also raises questions about predictive power.


\subsection{Modified Gravity Approaches}


An alternative interpretation holds that gravity itself behaves differently at galactic scales. Milgrom's Modified Newtonian Dynamics (MOND) successfully predicts galaxy rotation curves using a single acceleration scale a₀ ≈ 1.2 × 10⁻¹⁰ m/s². MOND's empirical success is remarkable: it predicts rotation curves from baryonic mass distributions alone, explaining correlations like the baryonic Tully-Fisher relation that ΛCDM must treat as emergent.


However, MOND faces significant challenges. It lacks a relativistic foundation, making gravitational lensing and cosmological predictions problematic. Relativistic extensions (TeVeS, BIMOND) introduce additional fields but face theoretical difficulties including superluminal propagation and instabilities. MOND also struggles with galaxy clusters, requiring either residual dark matter or modifications to the theory.


\subsection{Σ-Gravity: Coherent Torsion Enhancement}


Here we develop Σ-Gravity ("Sigma-Gravity"), grounded in teleparallel gravity—an equivalent reformulation of General Relativity (GR) where the gravitational field is carried by torsion rather than curvature. While mathematically equivalent to GR for classical predictions, teleparallel gravity suggests a different physical picture where gravity emerges from the parallel transport properties of spacetime.


\textbf{The central idea of Σ-Gravity:} In extended mass distributions with coherent motion—such as galactic disks with ordered circular rotation—torsion modes from spatially separated mass elements can interfere constructively. This \textbf{coherent superposition} produces measurable gravitational enhancement (Σ > 1) in dynamically cold systems while remaining undetectable in compact environments like the Solar System. The enhancement factor Σ gives the theory its name.


This mechanism naturally explains:


\begin{enumerate}
\item \textbf{Why enhancement appears at galactic scales:} Extended, ordered mass distributions allow torsion coherence
\item \textbf{Why the Solar System shows no anomaly:} Compact systems suppress coherence automatically
\item \textbf{Why a characteristic acceleration exists:} The cosmological horizon sets a fundamental decoherence scale
\item \textbf{Why clusters require larger enhancement:} Spherical geometry increases coherent mode counting
\end{enumerate}


\subsection{Summary of Results}


\begin{table}[h]
\centering
\begin{tabular}{lllll}
\toprule
Domain & Metric & Σ-Gravity & MOND & GR baryons \\
\midrule
SPARC galaxies (171) & RAR scatter & \textbf{0.100 dex} & 0.100 dex & 0.18–0.25 dex \\
SPARC head-to-head & Wins & \textbf{97} & 74 & — \\
MW rotation curve & RMS vs McGaugh & \textbf{5.7 km/s} & 2.1 km/s & 53.1 km/s \\
MW rotation curve & V(8 kpc) & \textbf{227.6 km/s}* & 233.0 km/s & 190.7 km/s \\
Galaxy clusters (N=2) & Preliminary & \textbf{2/2 in 68\%} & — & Baseline \\
Solar System & PPN γ−1 & \textbf{< 10⁻¹³} & < 10⁻⁵ & 0 \\
\bottomrule
\end{tabular}
\end{table}


\[
*Observed: 233.3 km/s (McGaugh/GRAVITY). Σ-Gravity: Δ = −5.7 km/s; MOND: Δ = −0.3 km/s.
\]


\medskip\hrule\medskip


\section{Theoretical Framework}


\subsection{Teleparallel Gravity as Mathematical Foundation}


In Einstein's General Relativity, gravity manifests as spacetime curvature described by the Riemann tensor. The Teleparallel Equivalent of General Relativity (TEGR) provides an alternative formulation where gravity is instead encoded in torsion—the antisymmetric part of an affine connection with vanishing curvature.


The fundamental dynamical variable in TEGR is the tetrad (vierbein) field e^a\_μ, which relates the spacetime metric to a local Minkowski frame:


\begin{equation}
g_{\mu\nu} = \eta_{ab} e^a_\mu e^b_\nu
\end{equation}


The torsion tensor is constructed from tetrad derivatives:


\begin{equation}
T^\lambda_{\mu\nu} = e^\lambda_a (\partial_\mu e^a_\nu - \partial_\nu e^a_\mu)
\end{equation}


The TEGR action produces field equations mathematically identical to Einstein's equations. The two formulations are related by a total derivative term, meaning TEGR makes identical predictions to GR for all classical tests.


\subsection{The Core Idea: Coherent Torsion Superposition}


\textbf{This is the central physical insight of Σ-Gravity.} The conceptual difference between GR and TEGR becomes significant when considering how torsion modes from extended sources combine.


In the path integral formulation of gravity, different geometric configurations contribute to the gravitational amplitude. For a compact source like the Sun, the classical saddle-point configuration dominates completely—quantum corrections are suppressed by factors of (ℓ\_Planck/r)² ≈ 10⁻⁶⁶, and torsion modes from different parts of the source add incoherently.


\textbf{However, for extended mass distributions with coherent motion—such as galactic disks with ordered circular rotation—the situation differs qualitatively.} Torsion contributions from spatially separated mass elements can interfere constructively when their phases remain aligned. This is directly analogous to:

\begin{itemize}
\item \textbf{Laser coherence:} Photons from different atoms add constructively when phase-locked
\item \textbf{Superconductivity:} Cooper pairs maintain phase coherence across macroscopic distances
\item \textbf{Antenna arrays:} Signals from multiple elements combine coherently to enhance gain
\end{itemize}


The effective gravitational field becomes:


\begin{equation}
g_{\rm eff}(\mathbf{x}) = g_{\rm bar}(\mathbf{x}) \cdot \Sigma(\mathbf{x})
\end{equation}


where g\_bar is the Newtonian/GR field from baryonic matter and \textbf{Σ ≥ 1 is the coherent enhancement factor} that gives the theory its name.


\textbf{Why coherence produces enhancement:} In teleparallel gravity, gravitational radiation carries two polarization modes (the same as in GR). In compact systems, typically one effective polarization aligned with the source-observer geometry contributes to measurements—torsion modes from different mass elements have random phases and average to zero for the perpendicular polarization.


In extended coherent systems, the ordered motion maintains phase alignment across the disk. Torsion modes from different mass elements can then add constructively, allowing both polarizations to contribute. Two independent modes adding in quadrature give:


\begin{equation}
\Sigma_{\rm baseline} = \sqrt{1^2 + 1^2} = \sqrt{2}
\end{equation}


Additional geometric factors from 3D integration over disk geometry increase this to √3 for galaxies and π√2 for spherical clusters. \textbf{The enhancement is not new physics beyond GR—it is GR's torsion formulation revealing effects that remain hidden in the curvature formulation when coherence conditions are met.}


\subsection{The Coherence Window}


Coherence requires sustained phase alignment among contributing torsion modes. Several physical mechanisms destroy coherence:


\begin{enumerate}
\item \textbf{Spatial separation:} Modes from distant regions accumulate phase mismatch
\item \textbf{Velocity dispersion:} Random stellar motions introduce phase noise
\item \textbf{Asymmetric structure:} Bars, bulges, and merger features disrupt ordered flow
\item \textbf{Differential rotation:} Spiral winding progressively misaligns initially coherent regions
\end{enumerate}


We model the coherence survival probability as:


\begin{equation}
W(r) = 1 - \left(\frac{\xi}{\xi + r}\right)^{n_{\rm coh}}
\end{equation}


\[
where ξ = (2/3)R_d is the coherence length scale and n_coh = 0.5 is the decay exponent derived from χ² decoherence statistics.
\]


\begin{figure}[h]
\centering
\includegraphics[width=0.8\textwidth]{../figures/coherence_window.png}
\caption{Figure: Coherence window}
\end{figure}


\textit{Figure 3: Left: Coherence window W(r) for different disk scale lengths. Right: Total enhancement Σ(r) as a function of radius at various accelerations, showing how coherence builds with radius.}


\subsection{Acceleration Dependence}


\[
The enhancement factor depends on the local baryonic gravitational acceleration g = g_bar through:
\]


\begin{equation}
h(g) = \sqrt{\frac{g^\dagger}{g}} \cdot \frac{g^\dagger}{g^\dagger + g}
\end{equation}


where g is the Newtonian acceleration from baryonic matter alone (not the total observed acceleration). This function produces flat rotation curves at low acceleration (g << g†) and recovers Newtonian/GR gravity at high acceleration (g >> g†).


\subsection{The Critical Acceleration Scale}


The critical acceleration is:


\begin{equation}
g^\dagger = \frac{cH_0}{2e} \approx 1.20 \times 10^{-10} \text{ m/s}^2
\end{equation}


where H₀ ≈ 70 km/s/Mpc is the Hubble constant and e = 2.718... is Euler's number. This matches the empirical MOND scale a₀ to within \textbf{0.4\%}, providing a physical explanation for the long-standing "MOND coincidence" that a₀ ~ cH₀.


\subsection{Unified Formula}


The complete enhancement factor is:


\begin{equation}
\boxed{\Sigma = 1 + A \cdot W(r) \cdot h(g)}
\end{equation}


with components:

\begin{itemize}
\item \textbf{h(g) = √(g†/g) × g†/(g†+g)} — universal acceleration function (g is baryonic acceleration)
\item \textbf{W(r) = 1 - (ξ/(ξ+r))^0.5} with ξ = (2/3)R\_d — coherence window (see SI §4 for derivation)
\item \textbf{A\_galaxy = √3 ≈ 1.73} — amplitude for disk galaxies (from 2D coherent mode geometry; see SI §5)
\item \textbf{A\_cluster = π√2 ≈ 4.44} — amplitude for spherical clusters (3D geometry; see SI §6)
\end{itemize}


The cluster/galaxy amplitude ratio π√2/√3 ≈ 2.57 matches the empirically observed ratio of 2.60 to within 1.2\%. This agreement is intriguing but warrants caution: the geometric derivations involve approximations whose robustness requires further investigation.


\subsection{Why This Formula (Not MOND's)}


\[
MOND's success with a₀ ≈ 1.2×10⁻¹⁰ m/s² has been known for 40 years, but lacked physical explanation. Σ-Gravity derives g† = cH₀/(2e) from cosmological horizon physics—matching a₀ to 0.4%—while the h(g) function emerges from teleparallel coherence, not phenomenological fitting.
\]


The two approaches produce similar curves but differ by ~7\% in the transition regime, making them experimentally distinguishable with high-precision rotation curves.


\begin{figure}[h]
\centering
\includegraphics[width=0.8\textwidth]{../figures/h_function_comparison.png}
\caption{Figure: h(g) function comparison}
\end{figure}


\textit{Figure 1: Enhancement functions h(g) for Σ-Gravity (derived from teleparallel coherence) vs MOND (empirical). The functions are similar but distinguishable.}


\subsection{Solar System Safety}


In compact systems, the coherence window W(r) → 0 and the enhancement automatically vanishes. The enhancement is suppressed by \textbf{8+ orders of magnitude} below current observational limits. This is not fine-tuning but an automatic consequence of the coherence mechanism: \textbf{compact systems cannot sustain the extended, ordered mass distributions required for torsion coherence.}


\begin{figure}[h]
\centering
\includegraphics[width=0.8\textwidth]{../figures/solar_system_safety.png}
\caption{Figure: Solar System safety}
\end{figure}


\textit{Figure 2: Enhancement (Σ-1) as a function of distance from the Sun. At planetary scales, the enhancement is < 10⁻¹⁴, far below observational bounds. Coherence mechanism automatically suppresses enhancement in compact systems.}


\medskip\hrule\medskip


\section{Results}


\subsection{Radial Acceleration Relation (SPARC Galaxies)}


We test the framework on the SPARC database (Lelli+ 2016) containing 175 late-type galaxies with high-quality rotation curves and 3.6μm photometry.


\textbf{Methodology:}

\begin{itemize}
\item \textbf{Mass-to-light ratio:} We adopt M/L = 0.5 M☉/L☉ at 3.6μm, the universal value recommended by Lelli+ (2016) based on stellar population models. This is not fitted per-galaxy.
\item \textbf{Distances and inclinations:} Fixed to SPARC published values; not varied in our analysis.
\item \textbf{Scatter metric:} RAR scatter is computed as the RMS of log₁₀(g\_obs/g\_pred) across all radial points.
\item \textbf{"No free parameters":} The Σ-Gravity formula uses A = √3 and g† = cH₀/(2e) derived from theory. The only external input is the universal M/L from stellar population models.
\end{itemize}


\textbf{Results (171 galaxies):}


\begin{table}[h]
\centering
\begin{tabular}{lll}
\toprule
Metric & Σ-Gravity & MOND \\
\midrule
Mean RAR scatter & \textbf{0.100 dex} & 0.100 dex \\
Median RAR scatter & \textbf{0.087 dex} & 0.085 dex \\
Head-to-head wins & \textbf{97 galaxies} & 74 galaxies \\
\bottomrule
\end{tabular}
\end{table}


Both theories achieve comparable overall scatter. Σ-Gravity wins on more individual galaxies (97 vs 74) when comparing per-galaxy RAR residuals, though this margin is not statistically overwhelming.


\begin{figure}[h]
\centering
\includegraphics[width=0.8\textwidth]{../figures/rar_derived_formula.png}
\caption{Figure: RAR plot}
\end{figure}


\textit{Figure 4: Radial Acceleration Relation for SPARC galaxies using derived formula. Gray points: observed accelerations. Blue line: Σ-Gravity prediction with A = √3. Red dashed: MOND.}


\begin{figure}[h]
\centering
\includegraphics[width=0.8\textwidth]{../figures/rc_gallery_derived.png}
\caption{Figure: Rotation curve gallery}
\end{figure}


\textit{Figure 5: Rotation curves for six representative SPARC galaxies selected for RAR scatter near the mean (0.100 dex). Black points: observed data. Green dashed: baryonic (GR). Blue solid: Σ-Gravity. Red dotted: MOND.}


\subsection{Milky Way Validation}


\[
We test the derived formula against the Milky Way rotation curve using McGaugh/GRAVITY data (HI terminal velocities + GRAVITY Θ₀ = 233.3 km/s at R₀ = 8 kpc). We adopt McGaugh's baryonic model (M* = 6.16×10¹⁰ M☉, giving V_bar ≈ 190 km/s at R=8 kpc).
\]


\textbf{Rotation curve comparison (5-15 kpc):}


\begin{table}[h]
\centering
\begin{tabular}{llll}
\toprule
Model & RMS vs McGaugh & V(8 kpc) & Δ at solar circle \\
\midrule
GR (baryons only) & 53.1 km/s & 190.7 km/s & −42.6 km/s \\
\textbf{Σ-Gravity} & \textbf{5.7 km/s} & \textbf{227.6 km/s} & \textbf{−5.7 km/s} \\
MOND & 2.1 km/s & 233.0 km/s & −0.3 km/s \\
NFW Dark Matter & 2.8 km/s & 233.9 km/s & +0.6 km/s \\
\bottomrule
\end{tabular}
\end{table}


\textbf{At solar circle (R = 8 kpc):} McGaugh/GRAVITY observed = 233.3 km/s. Σ-Gravity predicts 227.6 km/s (Δ = −5.7 km/s). All modified gravity models match within ~3\%, while GR baryons alone underpredict by 43 km/s (18\%).


\textbf{Comparison note:} MOND (RMS = 2.1 km/s) and NFW dark matter (RMS = 2.8 km/s) achieve better fits than Σ-Gravity (RMS = 5.7 km/s). This is expected: McGaugh's baryonic model was developed in a MOND context. Σ-Gravity's result demonstrates consistency with MW kinematics using zero MW-specific tuning, but does not outperform existing models.


\begin{figure}[h]
\centering
\includegraphics[width=0.8\textwidth]{../figures/mw_comprehensive_comparison.png}
\caption{Figure: MW rotation curve}
\end{figure}


\textit{Figure 4b: Milky Way rotation curve comparison. Left: McGaugh/GRAVITY observed (black) vs model predictions. Right: Residuals. Σ-Gravity (blue) achieves RMS = 5.7 km/s using derived parameters (A=√3, g†=cH₀/2e). Baryonic model: McGaugh M} = 6.16×10¹⁰ M☉.*


\textbf{Caveats:} The baryonic model has systematic uncertainties in bar/bulge decomposition that could affect all predictions. The slight rising trend in Σ-Gravity predictions (227→230 km/s from 5→15 kpc) vs declining observations (238→221 km/s) represents a shape mismatch that warrants further investigation.


\subsection{Galaxy Cluster Strong Lensing (Preliminary)}


Galaxy clusters provide a third test domain through strong gravitational lensing.


\textbf{Illustrative examples (N=2):}

\begin{itemize}
\item Abell 2261: Predicted θ\_E within 68\% credible interval ✓
\item MACSJ1149.5+2223: Predicted θ\_E within 68\% credible interval ✓
\end{itemize}


\textbf{Important caveat:} A sample of two clusters provides only illustrative examples, not statistically meaningful validation. Robust constraints would require dozens of clusters with well-characterized baryonic mass distributions, careful treatment of line-of-sight structures, and comparison to standard ΛCDM and MOND predictions.


\begin{figure}[h]
\centering
\includegraphics[width=0.8\textwidth]{../figures/cluster_holdout_validation.png}
\caption{Figure: Cluster holdout validation}
\end{figure}


\textit{Figure 6: Preliminary cluster tests. Left: Predicted vs observed Einstein radii. Right: Normalized residuals. N=2 is insufficient for statistical validation but demonstrates the formula's applicability to cluster scales.}


\subsection{Cross-Domain Consistency}


\begin{table}[h]
\centering
\begin{tabular}{llll}
\toprule
Domain & Formula & Amplitude & Performance \\
\midrule
Disk galaxies (171) & Σ = 1 + A·W·h & √3 & 0.100 dex RAR scatter \\
Milky Way & same & √3 & RMS = 5.7 km/s (cf. MOND 2.1) \\
Galaxy clusters & same & π√2 & 2/2 illustrative examples \\
\bottomrule
\end{tabular}
\end{table}


The amplitude ratio emerges from geometric arguments (spherical vs disk coherence geometry) and matches observation to ~1\%. However, this agreement should be treated with caution pending more rigorous derivation.


\begin{figure}[h]
\centering
\includegraphics[width=0.8\textwidth]{../figures/amplitude_comparison.png}
\caption{Figure: Amplitude comparison}
\end{figure}


\textit{Figure 7: Derived vs observed amplitudes. Galaxy amplitude √3 and cluster amplitude π√2 emerge from coherence geometry.}


\medskip\hrule\medskip


\section{Discussion}


\subsection{Relation to Dark Matter and MOND}


\textbf{Unlike particle dark matter:}

\begin{itemize}
\item No per-system halo fitting required (vs 2-3 parameters per galaxy in ΛCDM)
\item Naturally explains tight RAR scatter (emerges from universal coherence formula)
\item No invisible mass—only baryons contribute, coherently enhanced
\end{itemize}


\textbf{Unlike MOND:}

\begin{itemize}
\item \textbf{Physical mechanism identified:} coherent torsion superposition
\item Embedded in relativistic field theory (teleparallel gravity)
\item Automatic Solar System safety (coherence window, not hand-tuned)
\item Natural cluster/galaxy amplitude ratio (from coherence geometry)
\item Critical acceleration g† = cH₀/(2e) derived, not fitted
\end{itemize}


\subsection{Testable Predictions}


\begin{enumerate}
\item \textbf{Counter-rotating disks:} Reduced enhancement (coherence disrupted)
\item \textbf{Tidal streams:} Enhanced self-gravity in dynamically cold streams
\item \textbf{High-redshift galaxies:} Different dynamics if enhancement depends on coherence
\item \textbf{Transition regime shape:} Small but measurable differences from MOND in galaxies with g ~ g†
\end{enumerate}


\subsection{Limitations and Future Work}


\begin{itemize}
\item No Lagrangian formulation yet—enhancement mechanism is motivated but not derived from action principle
\item Cosmological predictions (CMB, structure formation) require additional development
\item Gravitational wave propagation in enhanced regime needs investigation
\end{itemize}


\medskip\hrule\medskip


\section{Methods}


\subsection{Unified Formula Implementation}


```python

\# Physical constants

c = 2.998e8          \# m/s

\[
H0_SI = 2.27e-18     # s⁻¹ (70 km/s/Mpc)
\]

\[
g_dagger = c * H0_SI / (2 * np.e)  # Critical acceleration
\]


def h\_universal(g):

    """Acceleration function h(g)"""

    return np.sqrt(g\_dagger / g) * g\_dagger / (g\_dagger + g)


def W\_coherence(r, R\_d):

    """Coherence window W(r)"""

\[
xi = (2/3) * R_d
\]

    return 1 - (xi / (xi + r)) ** 0.5


def Sigma(r, g\_bar, R\_d, A):

    """Enhancement factor"""

    return 1 + A \textit{ W\_coherence(r, R\_d) } h\_universal(g\_bar)

```


\medskip\hrule\medskip


\section{Code Availability}


Complete code repository: https://github.com/lrspeiser/SigmaGravity


\textbf{Key reproduction commands:}

```bash

\# SPARC holdout validation

python derivations/connections/validate\_holdout.py


\# Generate paper figures  

python scripts/generate\_paper\_figures.py


\# Milky Way zero-shot analysis

python scripts/analyze\_mw\_rar\_starlevel.py

```


All results use seed = 42 for reproducibility.


\medskip\hrule\medskip


\section{Supplementary Information}


Extended derivations, additional validation tests, parameter derivation details, morphology dependence analysis, gate derivations, cluster analysis details, and complete reproduction instructions are provided in SUPPLEMENTARY\_INFORMATION.md.


\medskip\hrule\medskip


\section{Figure Legends}


\textbf{Figure 1:} Enhancement function h(g) comparison showing ~7\% testable difference from MOND.


\textbf{Figure 2:} Solar System safety—coherence mechanism automatically suppresses enhancement.


\textbf{Figure 3:} Coherence window W(r) and total enhancement Σ(r).


\textbf{Figure 4:} Radial Acceleration Relation for SPARC galaxies with derived formula.


\textbf{Figure 4b:} Milky Way rotation curve comparing Σ-Gravity and MOND predictions to Eilers+ 2019 observations.


\textbf{Figure 5:} Rotation curve gallery for representative SPARC galaxies.


\textbf{Figure 6:} Cluster holdout validation with 2/2 coverage.


\textbf{Figure 7:} Amplitude comparison: √3 (galaxies) vs π√2 (clusters) from coherence geometry.


\end{document}
